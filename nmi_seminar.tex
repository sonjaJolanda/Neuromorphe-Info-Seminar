\documentclass{SeminarV2}
\usepackage[dvips]{graphicx}
\usepackage[utf8]{inputenc}
\usepackage{amssymb,amsmath,array}

\usepackage{xcolor} % added by me

%***********************************************************************
% !!!! IMPORTANT NOTICE ON TEXT MARGINS !!!!!
%***********************************************************************
%
% Please avoid using DVI2PDF or PS2PDF converters: some undesired
% shifting/scaling may occur when using these programs
% It is strongly recommended to use the DVIPS converters.
%
% Check that you have set the paper size to A4 (and NOT to letter) in your
% dvi2ps converter, in Adobe Acrobat if you use it, and in any printer driver
% that you could use.  You also have to disable the 'scale to fit paper' option
% of your printer driver.
%
% In any case, please check carefully that the final size of the top and
% bottom margins is 5.2 cm and of the left and right margins is 4.4 cm.
% It is your responsibility to verify this important requirement.  If these margin requirements and not fulfilled at the end of your file generation process, please use the following commands to correct them.  Otherwise, please do not modify these commands.
%
\voffset 0 cm \hoffset 0 cm \addtolength{\textwidth}{0cm}
\addtolength{\textheight}{0cm}\addtolength{\leftmargin}{0cm}

%***********************************************************************
% !!!! USE OF THE SeminarV2 LaTeX STYLE FILE !!!!!
%***********************************************************************
%
% Some commands are inserted in the following .tex example file.  Therefore to
% set up your Seminar submission, please use this file and modify it to insert
% your text, rather than staring from a blank .tex file.  In this way, you will
% have the commands inserted in the right place.

% Edited by Martin Bogdan.

\begin{document}
%style file for Seminar manuscripts
\title{Steifigkeit von Prothesenfüßen für Unterschenkelamputierte}

%***********************************************************************
% AUTHORS INFORMATION AREA
%***********************************************************************
\author{Sonja Klein
%
% DO NOT MODIFY THE FOLLOWING '\vspace' ARGUMENT
\vspace{.3cm}\\
%
% Addresses and institutions (remove "1- " in case of a single institution)
Universität Leipzig - Fakultät für Mathematik und Informatik \\
Augustusplatz 10, 04109 Leipzig - Deutschland
}
%***********************************************************************
% END OF AUTHORS INFORMATION AREA
%***********************************************************************

\maketitle

\begin{abstract}
Die richtige Steifigkeit bei lower limb Prothesen ist von entscheidender Bedeutung für die Wiederherstellung der Gehfähigkeit und Balance bei Menschen mit Unterschenkelamputation. Eine suboptimale Steifigkeit kann zu Kompensationen und langfristigen Gesundheitsproblemen führen. Die Bestimmung der optimalen Prothesen-Sprunggelenk-Steifigkeit für jeden Nutzer bleibt komplex. Studien untersuchen die Auswirkungen verschiedener Steifigkeiten auf Biomechanik und vergleichen die Präferenzen von Orthopädietechnikern und Prothesenträgern, die oft differieren. Patientenpräferenzen sind konsistent und korrelieren mit Gelenksymmetrie. Bevorzugte Steifigkeit variiert stark nach Aktivität, was variable Prothesen erfordert. Maschinelles Lernen bietet Potential zur automatisierten Präferenzvorhersage. Ziel ist die Verbesserung der Prothesengestaltung zur Leistungsoptimierung.

% das muss ich noch umschreiben, das ist komplett generiert, außerdem hab ich da keine Quellenangabe!

% Type your 100 words abstract here. Please do not modify the style
% of the paper. In particular, keep the text offsets to zero when
% possible (see above in this `SeminarV2.tex' sample file). You may
% \emph{slightly} modify it when necessary, but strictly respecting
% the margin requirements is mandatory (see the instructions to
% authors for more details).
\end{abstract}
\section{Einleitung}
Nach einer Unterschenkelamputation soll der prothetische Fuß das fehlende Sprunggelenk ersetzen \cite{Louessard.2022}. Es ist wichtig, dass diese Prothese möglichst gut die Funktionalität eines anatomischen Fußes ersetzt \cite{Stevens.2018}, denn Personen mit Unterschenkelamputationen haben ein energieineffizienten Gang, eine eingeschränkte Mobilität und eine eingeschränkte Gangstabilität \cite{Major.2014}\cite{Vaca.2022}.
Die Prothese muss an die individuellen Bedürfnisse des Nutzers angepasst werden \cite{Stevens.2018}. Dabei spielt die Steifigkeit einer Prothese die Schlüsselrolle \cite{Shepherd.2020}. 
Sie variiert je nach Modell und Kategorie der Prothese und ist wichtig für das Gesamtverhalten des Prothesenfußes, denn sie bestimmt, wie Energie beim Aufprall auf den Boden absorbiert und zurückgegeben wird und wie der Fuß in der terminalen Standphase des Ganges Halt bietet \cite{Shepherd.2020}. Die individuell richtige Steifigkeit für einen Patienten ist entscheidend für seine Gesamtmobilität und möglicherweise langfristige Gesundheit \cite{Shepherd.2020}. 
Ein zu steifer Fuß kann zu einer erhöhten Stoßbelastung und einem verlängerten Fersenkontakt während des frühen Stands, einer Überstreckung des Knies während des terminalen Stands und einer verringerten Speicherung und Rückgabe von Energie führen \cite{Shepherd.2020}. Ein zu nachgiebiger Fuß kann zu einem auffälligen, hörbaren Fußaufschlag in der frühen Gangphase und dem Verlust der vorderen Stütze in der Gangendphase führen \cite{Shepherd.2020}. 
Ganganomalien und Asymmetrien können zu langfristigen Nebenwirkungen wie chronische Rückenschmerzen oder Osteoarthritis führen \cite{Shepherd.2020}.

% muss hier eine Zusammenfassung ud ein Ziel rein?

\section{Verschiedene Prothesentypen für Patienten mit Unterschenkelamputationen}  % hier noch die Überschrift aussagekräftiger machen#
Prothesen gibt es in unterschiedlichen Preisklassen und vielfältigen Designs \cite{Stevens.2018}.
Einfache Solid-ankle-cushion-heel (SACH)-Füße bestehen aus einem massiven Knöchelblock und einem kompressiven Material in der Ferse \cite{Stevens.2018}. Der single-axis-Fuß hat besitzt ein mechanisches Gelenk, um ein Sprunggelenk nachzubilden, der multi-axis-Fuß hat flexible Elemente und ermöglicht so eine gedämpfte Bewegung in allen Bewegungsebenen und mit dem flexible-keel-Fuß wird die Standphase noch durch flexible Elemente im Vorfuß verbessert \cite{Stevens.2018}. 
Außerdem gibt es Energy-storing-and-returning (ESAR)-Füße aus elastischen Materialien, die sich unter Belastung verformen und anschließend wieder in ihre Ursprungsform zurückkehren \cite{Stevens.2018}. Die während der Verformung gespeicherte Energie wird am Ende freigesetzt, um den Gangzyklus mit Energie zu versorgen \cite{Stevens.2018}. 
Der ESAR-Fuß kann die nichtlineare Form der Drehmoment-Winkel-Kurve während der Standphase des Gangs jedoch nicht angemessen nachahmen \cite{Shepherd.2017}. Außerdem ist seine Steifigkeit für das Gehen in der Ebene optimiert, nicht auf beispielweise das Auf- und Absteigen von Treppen oder Rampen oder das Gehen auf unebenem Gelände \cite{Shepherd.2017}. Diese Aktivitäten benötigen unterschiedliche Steifigkeiten \cite{Shepherd.2017}. Um die Steifigkeit anpassen zu können, gibt es quasi-passive Prothesen \cite{Shepherd.2017}.

% muss oder will ich hier einen Absatz?
Der Variable Stiffness Prosthetic Ankle (VSPA)-Fuß ist eine quasi-passive Prothese mit einer nichtlinearen, individuell anpassbaren Drehmoment-Winkel-Kurve, die ihre Gesamtsteifigkeit während der Nutzung in Echtzeit anpassen kann \cite{Shepherd.2017}. Sie kann die biomechanischen Eigenschaften des menschlichen Sprunggelenks genauer nachbilden \cite{Shepherd.2017}. Der Fuß nutzt eine Blattfeder variabler Länge, ein nockenbasiertes Getriebe \cite{Shetty.2022}. Ein kleiner, batteriebetriebener Motor kann die Blattfeder und somit die Steifigkeit während der Schwungphase des Ganges anpassen \cite{Shetty.2022}. % hier noch mehr schreiben (shepherd 2017?) über motorisierte Prothesen?
Es gibt außerdem Forschung zu motorisierten Prothesen \cite{Shetty.2022}. Diese versuchen die Defizite herkömmlicher Prothesen zu vermeiden, sind aber teuer, schwer und wenig robust \cite{Shetty.2022}. Quasi-passive Prothesen mit einem kleinen Motor und einer Batterie versuchen dagegen einen Kompromiss zwischen der Funktionalität komplexer Systeme und deren Nachteilen zu finden \cite{Shetty.2022}. 

% zu viele Wörter in bold?

% Erläutern Sie die Rolle der mechanischen Eigenschaften von Prothesenfüßen, insbesondere der Sprunggelenksteifigkeit, für die Gangqualität und die Leistung des Nutzers. Stellen Sie das Problem dar, dass die meisten handelsüblichen Prothesenfüße eine feste Steifigkeit haben, obwohl die biomechanischen Anforderungen je nach Aktivität (Gehen auf unterschiedlichen Oberflächen, Steigungen, Treppen) variieren. Formulieren Sie klar das Ziel Ihrer Arbeit: Aktuelle Forschungsergebnisse zur bevorzugten Sprunggelenksteifigkeit bei Unterschenkelprothesen zusammenzufassen und deren Implikationen für Design und Anpassung von Prothesen, insbesondere verstellbaren Systemen, aufzuzeigen.

\section{Forschung zur Steifigkeit von Prothesenfüßen} 
Es ist wichtig das Design von Prothesen weiter zu erforschen um zu Verstehen wie Prothesen gestaltet werden können, sodass die negativen Auswirkungen einer Unterschenkelamputation möglichst gering bleiben \cite{Major.2014}.

\subsection{Auswirkungen verschiedener Prothesensteifigkeiten auf das Gleichgewicht und die Gangbiomechanik}
Im Rahmen einer Studie wurde die Stand- und Gehfähigkeit von Patienten mit Unterschenkelamputationen mit unterschiedlichen Steifigkeiten im Prothesengelenk untersucht \cite{Vaca.2022}. Zur Durchführung der Studie wurde ein Venture-Fuß verwendet, der drei verschiedenen Steifigkeiten durch den Austausch spezifischer Elemente ermöglicht \cite{Vaca.2022}. Die Ergebnisse deuteten auf eine verbesserte Balance als Folge einer höheren Steifigkeit hin \cite{Vaca.2022}. Die Variation der Steifigkeit hatte außerdem signifikante Auswirkungen auf verschiedene Aspekte der Gangbiomechanik, wie die Schrittlänge, die Schrittlängenasymmetrie, die Kinematik und Kinetik und die vertikalen Bodenreaktionskräfte \cite{Vaca.2022}. Dieser Erkenntnisse unterstreichen die Notwendigkeit einer Optimierung der Prothesensteifigkeit für die individuellen Bedürfnisse der Patienten \cite{Vaca.2022}.

% \subsection{XXX}
% CFAM erklären, falls ich das paper dazu drinnen lasse
% % was ist das Ergebnis dieses Papers gewesen? 
% 2014 wurden die Auswirkungen unterschiedlicher Rotationssteifigkeiten im Sprunggelenk einer experimentellen Fuß-Knöchel-Gelenkprothese auf verschiedene Gehparameter untersucht \cite{Major.2014}.
% Es wurde eine individuell angefertigte experimentelle Prothese, eine CFAM, verwendet, die eine systematische Variation mechanischer Eigenschaften ermöglichte, während andere Eigenschaften konstant bleiben konnten \cite{Major.2014}. 
% Die Ergebnisse der Studie deuten darauf hin, dass die Gehleistung bei Prothesen mit geringerer Dorsalextension (Hebung des Fußes in Richtung Schienbein) Steifigkeit profitieren könnte \cite{Major.2014}. % kann ich das in Klammern lassen?
% Die geringere Dorsalextension-Steifigkeit führte zu einer größeren maximalen Dorsalextension des Sprunggelenks und zu einer größeren maximalen Beugung des Kniegelenks der gesunden Seite im Stehen \cite{Major.2014}. Außerdem führte sie tendenziell zu einer reduzierten maximalen Bodenreaktionskraft während des Gehens, was potenziell vorteilhaft für die Gelenke und das Restglied sein könnte \cite{Major.2014}.  Zudem führte die geringere Dorsalextension-Steifigkeit zu einem geringeren Energierverbrauch beim Gehen, möglicherweise jedoch ohne klinische Signifikant \cite{Major.2014}.

\subsection{Wahrnehmung der optimalen Prothesensteifigkeit} % von Orthopädietechnikern und Prothesenträgern
Eine weitere Studie konzentriert sich auf die Analyse der Wahrnehmungen von Prothesenträgern und Orthopädietechnikern \cite{Shepherd.2020}. Forscher stützen ihre Empfehlungen auf biometrische Analysen, während Orthopädietechniker auf qualitatives Feedback von Patienten und ihre visuelle Beurteilung des Ganges zurückgreifen \cite{Shepherd.2020}. Eine qualitative hochwertige Analyse dieser Wahrnehmungen ist wichtig \cite{Shepherd.2020}. Patienten sind in der Lage Faktoren wie Komfort der Fußes, Leichtgängigkeit der Bewegung, Vertrauen in das Gleichgewicht oder auch die lokale Muskelermüdung wahrzunehmen \cite{Shepherd.2020}. Orthopädietechniker können dieses Feedback mit ihrer visuellen Einschätzung des Ganges des Patienten und jahrelanger Erfahrung kombinieren \cite{Shepherd.2020}. Die Autoren ließen Patienten mit Unterschenkelamputationen mit einer Prothese variabler Steifigkeit gehen, während Orthopädietechniker ihren Gang beobachteten \cite{Shepherd.2020}. Beide Gruppen gaben danach unabhängig voneinander ihre bevorzugte Steifigkeit an \cite{Shepherd.2020}. 
Patienten und Orthopädietechniker hatten in der Regel nicht die gleiche Präferenz \cite{Shepherd.2020}.  Orthopädietechniker bevorzugten eine höhere Steifigkeit als die Prothesenträger, genauer gesagt um plus 26\% \cite{Shepherd.2020}. Das könnte darauf hindeuten, dass die Anzeichen für eine zu geringe Steifigkeit visuell deutlich sind \cite{Shepherd.2020}. 
Die Meinungen der Patienten waren erheblich konsistenter, als die der verschiedenen Orthopädietechniker \cite{Shepherd.2020}. Auch wenn die Studie die optimale Steifigkeit nicht bestimmen kann, kann aus dieser Konsistenz abgeleitet werden, dass das Feedback der Patienten, wenn sie unterschiedliche Steifigkeiten testen können, sehr präzise ist \cite{Shepherd.2020}.

\subsection{Präferenz der Prothesensteifigkeit und ihre Beziehung zu messbaren Parametern}
Auch Clites et al. erkannten die Relevanz von Patientenpräferenzen für die Ermittlung der optimalen Prothesensteifigkeit und führten 2021 eine Studie durch, um die Beziehungen zwischen der Präferenz der Patienten für die Prothesensteifigkeit und verschiedenen anthropometrischen, metabolischen, biomechanischen und leistungsbasierten Messgrößen zu untersuchen \cite{Clites.2021}. Die Präferenzen eines Patienten folgen aus dessen physiologischen und biomechanischen Wahrnehmungen, einer Vielzahl an messbaren Informationen \cite{Clites.2021}.
Während der Nutzung eines VSPA-Fußes, der eine variable Steifigkeit während des Gehens ermöglicht, wurde die Gangbiomechanik und der metabolischer Energieverbrauch auf einem Laufband analysiert \cite{Clites.2021}.
Die Autoren konnten eine signifikante nichtlineare Beziehung zwischen Laufbandgeschwindigkeit und bevorzugter Steifigkeit entdecken \cite{Clites.2021}. Die bevorzugte Steifigkeit war am niedrigsten bei mittleren Geschwindigkeiten und stieg bei langsamerer und schnellerer Geschwindigkeit \cite{Clites.2021}. 
Schwerere Patienten bevorzugten außerdem keine höheren Steifigkeiten als leichtere Patienten \cite{Clites.2021}.
Es wurde kein signifikanter linearen Zusammenhang zwischen Gewicht und Steifigkeit gefunden \cite{Clites.2021}. 
Der Zusammenhang von Energieverbrauch und Steifigkeit wurde ebenfalls untersucht \cite{Clites.2021}. Auch hier konnte kein signifikanter linearer oder quadratischer Zusammenhang zwischen Steifigkeit und metabolischer Rate gefunden werden \cite{Clites.2021}. Die Laufbandgeschwindigkeit hatte jedoch einen signifikanten linearen Einfluss auf den Energieverbrauch \cite{Clites.2021}.
In der selbst gewählte Laufgeschwindigkeit konnten die Autoren ebenfalls einen Zusammenhang zur Steifigkeitspräferenz feststellen \cite{Clites.2021}. Die Patienten liefen langsamer, wenn die Steifigkeitswerte niedriger waren als bevorzugt \cite{Clites.2021}.
Auch biomechanische Merkmale, wie Gelenkwinkel und Gelenkmomente, wurden untersucht \cite{Clites.2021}. Dabei konnten zehn Merkmale ohne Zusammenhang, neun Merkmale mit linearem Zusammenhang und sechs Merkmale mit quadratischem Zusammenhang zur Steifigkeit gefunden werden \cite{Clites.2021}. Eines dieser Merkmale ist die Root-Mean-Square-Differenz des Sprunggelenkwinkels zwischen beiden Beinen, ein Maß für die Bewegungsasymmetrie des Sprunggelenks \cite{Clites.2021}. Dieses zeigte ein Extremum nahe der bevorzugten Steifigkeit \cite{Clites.2021}.
Daraus kann abgeleitet werden, dass ein Patient die Prothesensteifigkeit auswählt, bei der die Gangbewegung am symmetrischten ist \cite{Clites.2021}.
Patientenpräferenzen sind ein direktes Maß der Patientenwünsche, aber auch ein mit Unsicherheiten behaftetes Maß \cite{Clites.2021}. Die in der Studie identifizierten Korrelationen zwischen Präferenz und biomechanischen und leistungsbezogenen Maßen können genutzt werden, um schnell Design- und Steuerparameter von Fußprothesen auswählen zu können \cite{Clites.2021}.

\subsection{Bevorzugte Steifigkeit über verschiedene Aktivitäten hinweg}
In den meisten Studien zur Prothesensteifigkeit wird nur eine Aktivität, das Gehen, untersucht \cite{InstituteofElectricalandElectronicsEngineers.2024}. Eine Studie aus dem Jahr 2024 erforscht die bevorzugte Steifigkeit von Prothesenfüßen über fünf verschiedene Aktivitäten hinweg \cite{InstituteofElectricalandElectronicsEngineers.2024}. Auch hier wurde zur Durchführung der Studie der VSPA-Fuß verwendet \cite{InstituteofElectricalandElectronicsEngineers.2024}. Die vier Studienteilnehmer liefen mit der Prothese auf ebenerdigen Boden, bergauf und bergab auf eine Rampe und Treppen hoch und herunter \cite{InstituteofElectricalandElectronicsEngineers.2024}. Die bevorzugte Steifigkeit variierte stark zwischen den verschiedenen Aktivitäten \cite{InstituteofElectricalandElectronicsEngineers.2024}. Es gab eine durchschnittliche Abweichung von 31.8\% der bevorzugten Steifigkeit für ebenes Gehen \cite{InstituteofElectricalandElectronicsEngineers.2024}. Dies macht deutlich, dass Prothesen mit variabler Steifigkeit, die ihr mechanisches Verhalten den verschiedenen Aktivitäten anpassen können, notwendig sind, um die Anforderungen der Patienten zu erfüllen \cite{InstituteofElectricalandElectronicsEngineers.2024}.

\subsection{Das Vorhersagen von Steifigkeitspräferenzen einer Quasi-passiven Prothese}
Einige neue quasi-passive Prothesen können ihre Steifigkeit von Schritt zu Schritt anpassen um sie an unterschiedliche Aktivitäten, wie das Gehen oder das Treppensteigen, anzupassen \cite{Shetty.2022}. Die Steifigkeit muss dabei korrekt abgestimmt werden, um den gewünschten Effekt zu erreichen \cite{Shetty.2022}. Es konnte gezeigt werden, dass die bevorzugte Steifigkeit die kinematische Symmetrie im Sprunggelenk maximiert \cite{Shetty.2022}. Außerdem ist die Nutzerpräferenz eine wichtige Steuergröße \cite{Shetty.2022}. Patienten weisen unterschiedliche, aber konsistente Präferenzen für die Steifigkeit auf \cite{Shetty.2022}. Die Bestimmung der Patientenpräferenzen ist jedoch aufwendig und aus diesem Grund möchten Shetty et al. diese Präferenzen vorhersagen, anstatt sie experimentell zu erheben \cite{Shetty.2022}. 

Diese Vorhersagen versuchen die Autoren mit Hilfe eines maschinellen Lernmodells zu generieren \cite{Shetty.2022}. Dieses soll aus biomechanischen Daten Merkmale herausfiltern, aus denen Nutzerpräferenzen abgeleitet werden können \cite{Shetty.2022}. 
Die Autoren implementierten zwei klassische maschinelle Lernverfahren (ML) und drei Deep Learning (DL) Verfahren und verglichen sie miteinander \cite{Shetty.2022}. 
Sie analysierten zudem wie sich das Einbeziehen von subjektspezifischen Daten auf die Steifigkeitsvorhersage auswirkt \cite{Shetty.2022}. Und sie verglichen drei Gruppen von biomechanischen Signalen um zu verstehen wie sich die Menge und Art der Daten auf die Schätzgenauigkeit auswirken \cite{Shetty.2022}. 
In der Studie gingen sieben Versuchspersonen mit Unterschenkelamputation mit Hilfe eines VSPA-Fußes, der eine variable Steifigkeit ermöglicht, auf einem Laufband \cite{Shetty.2022}. Dabei ermittelten die Patienten die bevorzugte Steifigkeit mit Hilfe eines Drehknopfes selbstständig, bevor biomechanische Daten für unterschiedliche Steifigkeit- und Geschwindigkeitskombinationen gesammelt wurden \cite{Shetty.2022}. Die Versuchspersonen erarbeiteten außerdem in ihrer bevorzugten, einer schnelleren und einer langsameren Geschwindigkeit ihre bevorzugte Steifigkeit \cite{Shetty.2022}. Zur Erfassung der biomechanischen Daten wurde eine optische Bewegungserfassung und eine Kraftmessplatte verwendet \cite{Shetty.2022}.

Die biomechanischen Daten wurden mit OpenSim nachbearbeitet und analysiert bevor die Algorithmen angewendet wurden \cite{Shetty.2022}. Die Daten wurden außerdem, neben der Aufteilung in Trainingsdaten, Validierungsdaten und Testdaten, auch in Subjekt-unabhängige Daten und Subjekt-abhängige Daten unterteilt \cite{Shetty.2022}. Mit Hilfe einer Gittersuche und der Validierungsmenge wurden die Hyperparameter aller Algorithmen vor dem Training abgestimmt, um die Quadratwurzel des mittleren quadratischen Fehlers (RMSE) zwischen der vorhergesagten und der tatsächlichen bevorzugten Steifigkeit zu minimieren \cite{Shetty.2022}.
Es wurden zwei klassische ML-Algorithmen implementiert \cite{Shetty.2022}. Der K-Nearest Neighbour-Algorithmus (KNN) macht Vorhersagen basierend auf den Ähnlichkeiten der Daten, genauer gesagt auf Grundlage des gewichteten Durchschnitts der Abstandsfunktion für die K nächsten Punkte \cite{Shetty.2022}. Es wurde die euklidische Abstandsfunktion und ein Nachbarschaftswert von K=5 gewählt \cite{Shetty.2022}. Die Support Vector Regression (SVR) bildet die Daten auf eine höhere Dimension ab und sucht dort eine Hyperebene mit optimalen Rändern \cite{Shetty.2022}. Außerdem wurden drei DL-Algorithmen implementiert \cite{Shetty.2022}. bei allen drei DL-Algorithmen wurde die ReLU-Aktivierungsfunktion verwendet \cite{Shetty.2022}. Ein künstliches neuronales Netz (ANN) aus mehreren vollständig verbundenen Schichten ist der erste implementierte DL-Algorithmus \cite{Shetty.2022}. Für den zweiten Algorithmus, ein Convolutional Neural Network (CNN), wurden die Eingabedaten in ein 3D-Format umgewandelt \cite{Shetty.2022}. Der dritte Algorithmus ist ein Long Short-Term Memory (LSTM) Netzwerk \cite{Shetty.2022}. 

Die Algorithmen hatten einen signifikanten Einfluss auf den RMSE der Vorhersagen \cite{Shetty.2022}. Der RMSE mit den nutzerspezifischen Daten war um 67\% niedriger. Die drei DL-Algorithmen hatten einen niedrigeren RMSE als die beiden klassichen ML-Algorithmen, die DL-Algorithmen waren also besser \cite{Shetty.2022}. Der LSTM hatte den niedrigsten RMSE, $13.4\% \pm7.9\%$ und KNN hatte den höchsten RMSE mit $18.3\% \pm6.0\%$ \cite{Shetty.2022}. Die durchschnittliche Zeit zum Generieren der Vorhersage der Methoden dieser Arbeit war $1.99 \pm 2.22 ms$ \cite{Shetty.2022}.

Das Ergebnis dieser Studie ist die Erkenntnis, dass biomechanische Daten mit Deep Learning Modellen effektiv genutzt werden können, um Nutzerpräferenzen mit Trainingsdaten vorhersagen zu können \cite{Shetty.2022}. Das Einbeziehen nutzerspezifischer Trainingsdaten verbesserte die Schätzungen der bevorzugten Steifigkeit des Nutzers signifikant \cite{Shetty.2022}. Damit ergibt sich eine neue, einfachere und praktikablere Methode um die bevorzugte individuelle Prothesensteifigkeit von Patienten zu ermitteln \cite{Shetty.2022}. Die Ansätze dieser Arbeit sind vielversprechend für das Design und die Abstimmung von Roboterprothesen, aber durch den Zeitaufwand für die Vorhersagen limitiert \cite{Shetty.2022}. Zukünftige Arbeiten mit größeren Datensätzen könnten die Leistung und die Generalisierbarkeit der Modelle verbessern \cite{Shetty.2022}. Außerdem ist diese Studie auf die Aktivität Gehen beschränkt und die Untersuchung anderer Aktivitäten ist noch offen \cite{Shetty.2022}. 

% auch noch das Paper machen?: Armannsd: file:///C:/Users/sonja/IdeaProjects/Neuromorphe-Info-Seminar/Quellen/1-s2.0-S0268003321002060-main.pdf
% Task dependent changes in mechanical and biomechanical measures result from manipulating stiffness settings in a prosthetic foot 
\section{Schlussfolgerung}
Die Ergebnisse der verschiedenen Studien zeigen, dass die richtige Steifigkeit für einzelne Patienten und Aktivitäten wichtig beim Design einer Prothese ist. Die individuelle Steifigkeit ist ein zentraler Faktor für die Mobilität und das Wohlbefinden von Patienten mit Unterschenkelamputation, und sie variiert je nach Aktivität. Ein spannender Ansatz ist das Vorhersagen der bevorzugten Steifigkeit mit Hilfe von maschinellen Lernverfahren mit denen Prothesen ihre Steifigkeit während des Gangs anpassen könnten. Es ist ein vielversprechender Ansatz, der allerdings noch viel Potenzial für weitere Forschung bietet. 

% in research rabbit gab es noch drei Paper die dieses Paper zitieren, die könnte ich mir an sich auch noch ansehen 

% put my opinion in the paper
% es hieß doch irgendwie mal man soll nicht in der Vergangenheit schreiben, aber hier macht Vergangenheit ja irgendwie Sinn?
% am Ende noch Synonyme für das Wort Patienten finden 
% am Schluss noch mal mit der Vorlage von Bogdan vergleichen
% auch Rechtschreibung und Grammatikfehler prüfen lassen

% \section{Infos von Bogdan}
% \subsection{Page format and margins}
% Please avoid using DVI2PDF or PS2PDF converters: some undesired 
% shifting/scaling may occur when using these programs
% It is strongly recommended to use the DVIPS converters. 
% %
% Check that you have set the paper size to A4 (and NOT to letter) in your
% dvi2ps converter, in Adobe Acrobat if you use it, and in any printer driver
% that you could use.  You also have to disable the 'scale to fit paper' option
% of your printer driver.
% %
% In any case, please check carefully that the final size of the top and
% bottom margins is 5.2 cm and of the left and right margins is 4.4 cm.
% it is your responsibility to verify this important requirement.  If these margin requirements and not fulfilled at the end of your file generation process, please use the commands at the beginning of the SeminarV2.tex file to correct them.  Otherwise, please do not modify these commands.

% \subsection{Style information}
% Please do not add page numbers to this style; page numbers will be added by the publisher. Do not add headings to your document.
% \subsection{Mathematics}
% You may include additional packages for typesetting
% algorithms, mathematical formula or to define new operators and environments
% if and only if there is no conflict with the SeminarV2.cls
% file.

% It is recommended to avoid the numbering of equations when not
% necessary. When dealing with equation arrays, it could be
% necessary to label several (in)equalities. You can do it using the
% `$\backslash$stackrel' operator (see the SeminarV2.tex source file);
% example:

% \begin{eqnarray}
% c&=&|d|+|e|\nonumber\\
% &\stackrel{\text{(a)}}{=}&d+e\nonumber\\
% &\stackrel{\text{(b)}}{\geq}&\sqrt{f}\enspace,
% \end{eqnarray}
% \noindent where the equality (a) results from the fact that both
% $d$ and $e$ are positive while (b) comes from the definition of
% $f$.

% \subsection{Tables and figures}

% Figure \ref{Fig:MV} shows an example of figure and related
% caption.  Do not use too small symbols and lettering in your
% figures.  Warning: your paper will be printed in black and white
% in the proceedings.  You may insert color figures, but it is your
% responsibility to check that they print correctly in black and
% white.  The color version will be kept in the Seminar electronic
% proceedings available on the web.

% \begin{figure}[ht]
% \centering
% % \includegraphics[scale=0.5]{any_question.eps}
% \mycaption{Any questions?\label{Fig:MV}}
% \end{figure}

% Table \ref{Tab:AgeWeight} shows an example of table.

% \begin{table}[h!]
%   \centering
%   \begin{tabular}{|c|c|c|}
%     \hline
%     ID & age & weight \\
%     \hline
%     1& 15 & 65 \\
%     2& 24 & 74\\
%     3& 18 & 69 \\
%     4& 32 & 78 \\
%     \hline
%   \end{tabular}
%   \mycaption{Age and weight of people.\label{Tab:AgeWeight}}
% \end{table}

% ****************************************************************************
% BIBLIOGRAPHY AREA
% ****************************************************************************

\begin{footnotesize}
\bibliographystyle{unsrt}
\bibliography{own.bib}
\end{footnotesize}

% ****************************************************************************
% END OF BIBLIOGRAPHY AREA
% ****************************************************************************

\end{document}
