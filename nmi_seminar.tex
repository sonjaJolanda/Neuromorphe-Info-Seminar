\documentclass{SeminarV2}
\usepackage[dvips]{graphicx}
\usepackage[utf8]{inputenc}
\usepackage{amssymb,amsmath,array}

\usepackage{xcolor} % added by me

%***********************************************************************
% !!!! IMPORTANT NOTICE ON TEXT MARGINS !!!!!
%***********************************************************************
%
% Please avoid using DVI2PDF or PS2PDF converters: some undesired
% shifting/scaling may occur when using these programs
% It is strongly recommended to use the DVIPS converters.
%
% Check that you have set the paper size to A4 (and NOT to letter) in your
% dvi2ps converter, in Adobe Acrobat if you use it, and in any printer driver
% that you could use.  You also have to disable the 'scale to fit paper' option
% of your printer driver.
%
% In any case, please check carefully that the final size of the top and
% bottom margins is 5.2 cm and of the left and right margins is 4.4 cm.
% It is your responsibility to verify this important requirement.  If these margin requirements and not fulfilled at the end of your file generation process, please use the following commands to correct them.  Otherwise, please do not modify these commands.
%
\voffset 0 cm \hoffset 0 cm \addtolength{\textwidth}{0cm}
\addtolength{\textheight}{0cm}\addtolength{\leftmargin}{0cm}

%***********************************************************************
% !!!! USE OF THE SeminarV2 LaTeX STYLE FILE !!!!!
%***********************************************************************
%
% Some commands are inserted in the following .tex example file.  Therefore to
% set up your Seminar submission, please use this file and modify it to insert
% your text, rather than staring from a blank .tex file.  In this way, you will
% have the commands inserted in the right place.

% Edited by Martin Bogdan.

\begin{document}
%style file for Seminar manuscripts
\title{Steifigkeit von Prothesenfüßen für Unterschenkelamputierte}

%***********************************************************************
% AUTHORS INFORMATION AREA
%***********************************************************************
\author{Sonja Klein
%
% DO NOT MODIFY THE FOLLOWING '\vspace' ARGUMENT
\vspace{.3cm}\\
%
% Addresses and institutions (remove "1- " in case of a single institution)
Universität Leipzig - Fakultät für Mathematik und Informatik \\
Augustusplatz 10, 04109 Leipzig - Deutschland
}
%***********************************************************************
% END OF AUTHORS INFORMATION AREA
%***********************************************************************

\maketitle

\begin{abstract}
    Type your 100 words abstract here. Please do not modify the style
    of the paper. In particular, keep the text offsets to zero when
    possible (see above in this `SeminarV2.tex' sample file). You may
    \emph{slightly} modify it when necessary, but strictly respecting
    the margin requirements is mandatory (see the instructions to
    authors for more details).
\end{abstract}
\section{Einleitung}
Hintergrund: Herausforderungen bei Amputierten, Mängel aktueller Prothesen *   Aufkommende variable Prothesen als Lösungsansatz *   Forschungslücke: Präferierte Steifigkeit über verschiedene Aktivitäten hinweg *   Ziel der vorliegenden Studie

Inhalt: Beginnen Sie mit der Bedeutung von Unterschenkelprothesen für Menschen nach einer Amputation und den Herausforderungen, denen sie im Alltag begegnen (z.B. ineffizientes Gangbild, eingeschränkte Mobilität, Beschwerden). Erläutern Sie die Rolle der mechanischen Eigenschaften von Prothesenfüßen, insbesondere der Sprunggelenksteifigkeit, für die Gangqualität und die Leistung des Nutzers. Stellen Sie das Problem dar, dass die meisten handelsüblichen Prothesenfüße eine feste Steifigkeit haben, obwohl die biomechanischen Anforderungen je nach Aktivität (Gehen auf unterschiedlichen Oberflächen, Steigungen, Treppen) variieren. Formulieren Sie klar das Ziel Ihrer Arbeit: Aktuelle Forschungsergebnisse zur bevorzugten Sprunggelenksteifigkeit bei Unterschenkelprothesen zusammenzufassen und deren Implikationen für Design und Anpassung von Prothesen, insbesondere verstellbaren Systemen, aufzuzeigen.

\section{Hintergrund und Relevanz der Forschung}  % hier noch die Überschrift aussagekräftiger machen
Bei Personen mit Amputationen lässt sich ein energieineffizienter Gang und eine eingeschränkte Mobilität feststellen \cite{Major.2014}.
Es ist als wichtig, dass eine Prothese möglichst gut die Funktionalität eines anatomischen Fußes ersetzt \cite{Stevens.2018}. Dabei gibt es Prothesen in jeder Preisklasse und vielfältigem Design \cite{Stevens.2018}.

Einfache \textbf{Solid-ankle-cushion-heel (SACH)-Füße} bestehen aus einem massiven Knöchelblock und einem kompressiven Material in der Ferse \cite{Stevens.2018}. Der \textbf{single-axis-Fuß} hat besitzt ein mechanisches Gelenk, um ein Sprunggelenk nachzubilden, der \textbf{multi-axis-Fuß} hat flexible Elemente und ermöglicht so eine gedämpfte Bewegung in allen Bewegungsebenen und mit dem \textbf{flexible-keel-Fuß} wird die Standphase noch durch flexible Elemente im Vorfuß verbessert \cite{Stevens.2018}. 
Außerdem gibt es \textbf{Energy-storing-and-returning (ESAR)-Füße} aus elastischen Materialien, die sich unter Belastung verformen und anschließend wieder in ihre Ursprungsform zurückkehren, wobei die während der Verformung gespeicherte Energie am Ende freigesetzt wird, um den Gangzyklus mit Energie zu versorgen \cite{Stevens.2018}.

Bei dieser Vielfalt ist es wichtig, die Prothese an die Bedürfnisse des Nutzers anzupassen \cite{Stevens.2018}. Dabei spielt die Steifigkeit einer Prothese die Schlüsselrolle \cite{Shepherd.2020}. 
Sie variiert je nach Modell und Kategorie der Prothese und ist wichtig für das Gesamtverhalten des Prothesenfußes, denn sie bestimmt, wie Energie beim Aufprall auf den Boden absorbiert und zurückgegeben wird und wie der Fuß in der terminalen Standphase des Ganges Halt bietet \cite{Shepherd.2020}. Die individuell richtige Steifigkeit für einen Patienten ist entscheidend für seine Gesamtmobilität und möglicherweise langfristige Gesundheit \cite{Shepherd.2020}. 
Ein zu steifer Fuß kann zu einer erhöhten Stoßbelastung und einem verlängerten Fersenkontakt während des frühen Stands, einer Überstreckung des Knies während des terminalen Stands und einer verringerten Speicherung und Rückgabe von Energie führen \cite{Shepherd.2020}. 
Ein zu nachgiebiger Fuß kann zu einem Foot Slap in der frühen Gangphase und dem Verlust der vorderen Stütze in der Gangendphase führen \cite{Shepherd.2020}. 
Ganganomalien und Asymmetrien können zu langfristigen Nebenwirkungen wie chronische Rückenschmerzen oder Osteopenie führen \cite{Shepherd.2020}. \\

\section{Forschung zur Steifigkeit von Prothesenfüßen} 
Es ist also wichtig das Design von Prothesen weiter zu erforschen \cite{Major.2014}. \textcolor{red}{XX} untersuchten 2014 die Auswirkungen unterschiedlicher Rotationssteifigkeiten im Sprunggelenk einer experimentellen Fuß-Knöchel-Gelenkprothese auf verschiedene Gehparameter \cite{Major.2014}.  % TODO 
Es wurde eine individuell angefertigte experimentelle Prothese, eine CFAM, verwendet, die eine systematische Variation mechanischer Eigenschaften ermöglichte, während andere Eigenschaften konstant bleiben konnten \cite{Major.2014}. 
Die Ergebnisse der Studie deuten darauf hin, dass die Gehleistung bei Prothesen mit geringerer Dorsalextension (Hebung des Fußes in Richtung Schienbein) Steifigkeit profitieren könnte \cite{Major.2014}. % kann ich das in Klammern lassen?
Die geringere Dorsalextension-Steifigkeit führte zu einer größeren maximalen Dorsalextension des Sprunggelenks und zu einer größeren maximalen Beugung des Kniegelenks der gesunden Seite im Stehen \cite{Major.2014}. Außerdem führte sie tendenziell zu einer reduzierten maximalen Bodenreaktionskraft während des Gehens, was potenziell vorteilhaft für die Gelenke und das Restglied sein könnte \cite{Major.2014}.  Zudem führte die geringere Dorsalextension-Steifigkeit zu einem geringeren Energierverbrauch beim Gehen, möglicherweise jedoch ohne klinische Signifikant \cite{Major.2014}.

Eine weitere Studie von \textcolor{red}{XX} konzentrierte sich auf die Analyse der Wahrnehmungen von Prothesenträgern und Orthopädietechnikern \cite{Shepherd.2020}. Forscher stützen ihre Empfehlungen auf biometrische Analysen, während Orthopdietechniker sich auf qualitatives Feedback von Patienten und die visuelle Beurteilung des Ganges stützen \cite{Shepherd.2020}. Aus diesem Grund ist eine qualitative hochwertige Analyse der Wahrnehmung wichtig \cite{Shepherd.2020}. Patienten sind in der Lage Faktoren wie Komfort der Fußes, Leichtgängigkeit der Bewegung, Vertrauen in das Gleichgewicht oder auch die lokale Muskelermüdung wahrzunehmen \cite{Shepherd.2020}. Orthopädietechniker können sich neben diesem Feedback auch auf ihre visuelle Einschätzung des Ganges des Patienten und jahrelange Erfahrung stützen \cite{Shepherd.2020}. Die Studie betrachtet beide dieser Wahrnehmungen \cite{Shepherd.2020}. %Ergebnisse


\cite{Clites.2021} % dann das Paper auch kurz zusammenfassen

\cite{Louessard.2022} % dann das Paper auch kurz zusammenfassen 

\cite{Vaca.2022} % dann das Paper auch kurz zusammenfassen


\dots \\



Einschränkungen herkömmlicher Prothesen (Carbon-Verbundfedern mit fester Mechanik) *   Folgen für Nutzer (Unbehagen, kompensatorische Bewegungen, geringere Mobilität, sekundäre Erkrankungen) *   Notwendigkeit der Anpassung der Mechanik an unterschiedliche Aktivitäten *   Konzept variabler Steifigkeit bei Prothesen *   Benutzerpräferenz als potenzielles "Meta-Kriterium" zur intelligenten Anpassung *   Hinweise auf unterschiedliche präferierte Steifigkeit und kinematische/metabolische Vorteile durch Variation

Inhalt: 

Beschreiben Sie kurz die Funktion eines Prothesenfußes während des Gangzyklus und wie er versucht, die Rolle des biologischen Fußes und Sprunggelenks zu ersetzen (Stoßabsorption, Unterstützung im Stand, Abstoß). 

% DONE: Erklären Sie, dass die Sprunggelenksteifigkeit ein zentraler mechanischer Parameter ist, der dieses Verhalten beeinflusst. Gehen Sie darauf ein, dass eine ungeeignete Steifigkeit zu Gangabweichungen führen kann (z.B. Fusschlag, Knie-Hyperextension, verminderte Energiespeicherung), was langfristig zu sekundären Beschwerden (z.B. Schmerzen, Gelenkdegeneration) führen kann.

Erwähnen Sie, dass die Suche nach biomechanischen Markern für die "optimale" Steifigkeit bisher nicht eindeutig war, was die Bedeutung der Nutzerwahrnehmung und Präferenz hervorhebt. Stellen Sie die Idee verstellbarer Prothesen als potenziellen Weg vor, um die Steifigkeit an Nutzerbedürfnisse und Aktivitäten anzupassen.
\section{Forschung zur Steifigkeit von Prothesenfüßen} 
Es ist also wichtig das Design von Prothesen weiter zu erforschen \cite{Major.2014}.

\subsection{XXX}
% was ist das Ergebnis dieses Papers gewesen? 
\textcolor{red}{XTODOX} untersuchten 2014 die Auswirkungen unterschiedlicher Rotationssteifigkeiten im Sprunggelenk einer experimentellen Fuß-Knöchel-Gelenkprothese auf verschiedene Gehparameter \cite{Major.2014}.  % TODO 
Es wurde eine individuell angefertigte experimentelle Prothese, eine CFAM, verwendet, die eine systematische Variation mechanischer Eigenschaften ermöglichte, während andere Eigenschaften konstant bleiben konnten \cite{Major.2014}. 
Die Ergebnisse der Studie deuten darauf hin, dass die Gehleistung bei Prothesen mit geringerer Dorsalextension (Hebung des Fußes in Richtung Schienbein) Steifigkeit profitieren könnte \cite{Major.2014}. % kann ich das in Klammern lassen?
Die geringere Dorsalextension-Steifigkeit führte zu einer größeren maximalen Dorsalextension des Sprunggelenks und zu einer größeren maximalen Beugung des Kniegelenks der gesunden Seite im Stehen \cite{Major.2014}. Außerdem führte sie tendenziell zu einer reduzierten maximalen Bodenreaktionskraft während des Gehens, was potenziell vorteilhaft für die Gelenke und das Restglied sein könnte \cite{Major.2014}.  Zudem führte die geringere Dorsalextension-Steifigkeit zu einem geringeren Energierverbrauch beim Gehen, möglicherweise jedoch ohne klinische Signifikant \cite{Major.2014}.

\subsection{Wahrnehmung der optimalen Prothesensteifigkeit von Orthopädietechnikern und Prothesenträgern}
Eine weitere Studie von \textcolor{red}{XTODOX} aus dem Jahr 2020 konzentriert sich auf die Analyse der Wahrnehmungen von Prothesenträgern und Orthopädietechnikern \cite{Shepherd.2020}. Forscher stützen ihre Empfehlungen auf biometrische Analysen, während Orthopdietechniker sich auf qualitatives Feedback von Patienten und die visuelle Beurteilung des Ganges stützen \cite{Shepherd.2020}. Aus diesem Grund ist eine qualitative hochwertige Analyse der Wahrnehmung wichtig \cite{Shepherd.2020}. Patienten sind in der Lage Faktoren wie Komfort der Fußes, Leichtgängigkeit der Bewegung, Vertrauen in das Gleichgewicht oder auch die lokale Muskelermüdung wahrzunehmen \cite{Shepherd.2020}. Orthopädietechniker können sich neben diesem Feedback auch auf ihre visuelle Einschätzung des Ganges des Patienten und jahrelange Erfahrung stützen \cite{Shepherd.2020}. 
Patienten und Orthopdietechniker hatten in der Regel nicht die gleiche Präferenz \cite{Shepherd.2020}.  Orthopädietechniker bevorzugten eine höhere Steifigkeit als die Prothesenträger, genauer gesagt um plus 26\% \cite{Shepherd.2020}. Das könnte darauf hindeuten, dass die Anzeichen für eine zu geringe Steifigkeit visuell deutlich sind \cite{Shepherd.2020}. 
Die Meinungen der Patienten waren erheblich konsistenter, als die der verschiedenen Orthopädietechniker \cite{Shepherd.2020}. Auch wenn die Studie die optimale Steifigkeit nicht bestimmen kann, kann aus dieser Konsistenz jedoch abgeleitet werden, dass das Feedback der Patienten, wenn sie unterschiedliche Steifigkeiten testen können, sehr präzise ist \cite{Shepherd.2020}.

\subsection{Präferenz der Prothesensteifigkeit und ihre Beziehung zu messbaren Parametern}
Auch \textcolor{red}{XToDoX} erkannten die Relevanz von Patientenpräferenzen für die Prothesensteifigkeit und führten 2021 eine Studie durch, um die Beziehungen zwischen der Präferenz der Patienten für die Prothesensteifigkeit und verschiedenen anthropometrischen, metabolischen, biomechanischen und leistungsbasierten Messgrößen zu untersuchen \cite{Clites.2021}. Die Präferenzen eines Patienten folgen aus dessen physiologischen und biomechanischen Wahrnehmungen, einer Vielzahl an messbaren Informationen \cite{Clites.2021}.
Während der Nutzung des \textbf{Variable Stiffness Prosthetic Ankle (VSPA)}-Fußes, der eine variable Steifigkeit während des Gehens ermöglicht, wurden die Gangbiomechanik und der metabolischer Energieverbrauch auf einem Laufband analysiert \cite{Clites.2021}. \textcolor{red}{XToDoX} konnten einiges feststellen \cite{Clites.2021}. 
Es gibt eine signifikante nichtlineare Beziehung zwischen Laufbandgeschwindigkeit und bevorzugter Steifigkeit \cite{Clites.2021}. Die bevorzugte Steifigkeit ist am niedrigsten bei mittleren Geschwindigkeiten und steigt bei langsamerer und schnellerer Geschwindigkeit \cite{Clites.2021}. 
Schwerere Patienten bevorzugen außerdem keine höheren Steifigkeiten als leichtere Patienten \cite{Clites.2021}.
Es wurde kein signifikanter linearen Zusammenhang zwischen Gewicht und Steifigkeit gefunden \cite{Clites.2021}. 
Der Zusammenhang von Energierverbrauch und Steifigkeit wurde ebenfalls untersucht \cite{Clites.2021}. Auch hier konnte kein signifikanter linearer oder quadratischer Zusammenhang zwischen Steifigkeit und metabolischer Rate gefunden werden \cite{Clites.2021}. Die Laufbandgeschwindigkeit hatte jedoch einen signifikanten linearen Einfluss auf den Energieverbrauch \cite{Clites.2021}.
In der selbst gewählte Laufgeschwindigkeit konnten die Autoren ebenfalls einen Zusammenhang zur Steifigkeitspräferenz feststellen \cite{Clites.2021}. Die Patienten liefen langsamer, wenn die Steifigkeitswerte niedriger waren als bevorzugt \cite{Clites.2021}.
Auch biomechanische Merkmale, wie Gelenkwinkel und Gelenkmomente, wurden untersucht \cite{Clites.2021}. Dabei konnten zehn Merkmale ohne Zusammenhang, neun Merkmale mit linearem Zusammenhang und sechs Merkmale mit quadratischem Zusammenhang zur Steifigkeit gefunden werden \cite{Clites.2021}. Eines dieser Merkmale ist die Root-Mean-Square-Differenz des Sprunggelenkwinkels zwischen beiden Beinen, ein Maß für die Bewegungsasymmetrie des Sprunggelenks \cite{Clites.2021}. Dieses zeigte ein Extremum nahe der bevorzugten Steifigkeit \cite{Clites.2021}.
Daraus kann abgeleitet werden, dass ein Patient die Prothesensteifigkeit auswählt, bei der die Gangbewegung am symmetrischten ist \cite{Clites.2021}.
Patientenpräferenzen sind ein direktes Maß der Patientenwünsche, aber auch ein mit Unsicherheiten behaftetes Maß \cite{Clites.2021}. Die in der Studie identifizierten Korrelationen zwischen Präferenz und biomechanischen und leistungsbezogenen Maßen können genutzt werden, um schnell Design- und Steuerparameter von Fußprothesen auswählen zu können \cite{Clites.2021}.

\subsection{XX}
\cite{Louessard.2022} % dann das Paper auch kurz zusammenfassen 

\subsection{XX}
\cite{Vaca.2022} % dann das Paper auch kurz zusammenfassen

\subsection{XX}
\cite{InstituteofElectricalandElectronicsEngineers.2024} % dann das Paper auch kurz zusammenfassen: Präferierte Steifigkeit über verschiedene Aktivitäten hinweg

\subsection{Das Vorhersagen von Steifigkeitspräferenzen einer Quasi-passiven Prothese}






In den letzten Jahren gab es einige Fortschritte in der Entwicklung von Prothesen \cite{Shetty.2022}. Motorisierte Prothesen versuchen die Defizite herkömmlicher Prothesen zu vermeiden, sind aber teuer, schwer und wenig robust \cite{Shetty.2022}. Quasi-passive Prothesen mit einem kleinen Motor und einer Batterie versuchen einen Kompromiss zwischen der Funktionalität komplexer Systeme und deren Herausforderungen zu finden \cite{Shetty.2022}. 
Einige neue quasi-passive Prothesen können die Steifigkeit von Schritt zu Schritt anpassen um sie an unterschiedliche Aktivitäten, wie Gehen oder Treppensteigen, anzupassen \cite{Shetty.2022}. Die Steifigkeit muss dabei korrekt abgestimmt werden, um den gewünschten Effekt zu erreichen \cite{Shetty.2022}. Diese Aufgabe ist eine noch offene und wichtige Forschungsfrage \cite{Shetty.2022}. Das Konzept der Nutzerpräferenz setzt sich langsam als Steuergröße für assitive Technologien durch \cite{Shetty.2022}. Patienten weisen unterschiedliche, aber konsistente Präferenzen für die Steifigkeit auf \cite{Shetty.2022}. Außerdem konnte gezeigt werden, dass die bevorzugte Steifigkeit die kinematische Symmetrie im Sprunggelenk maximiert \cite{Shetty.2022}. Die Bestimmung der Patientenpräferenzen ist jedoch aufwendig \cite{Shetty.2022}. Aus diesem Grund möchten \textcolor{red}{XTODOX} diese Präferenzen vorhersagen, anstatt sie experimentell zu erheben \cite{Shetty.2022}. Diese Vorhersagen versuchen die Autoren mit Hilfe eines maschinellen Lernmodells zu generieren \cite{Shetty.2022}. Dieses soll aus biomechanischen Daten Merkmale herausfiltern, aus denen Nutzerpräferenzen abgeleitet werden können \cite{Shetty.2022}. 

Die Autoren implementierten zwei klassische maschinelle Lernverfahren und drei Deep Learning Verfahren und verglichen sie miteinander \cite{Shetty.2022}. % hier noch genauer beschreiben 

Das Ergebnis dieser Studie ist die Erkenntnis, dass biomechanische Daten mit Deep Learning Modellen effektiv genutzt werden können, um Nutzerpräferenzen mit nutzerspezifischen Trainingsdaten vorhersagen zu können \cite{Shetty.2022}. Damit ergibt sich eine neue, einfachere und praktikablere Methode um die bevorzugte individuelle Prothesensteifigkeit von Patienten zu ermitteln \cite{Shetty.2022}. 
\section{Diskussion}
% put my opinion in the paper


% \section{Infos von Bogdan}
% \subsection{Page format and margins}
% Please avoid using DVI2PDF or PS2PDF converters: some undesired 
% shifting/scaling may occur when using these programs
% It is strongly recommended to use the DVIPS converters. 
% %
% Check that you have set the paper size to A4 (and NOT to letter) in your
% dvi2ps converter, in Adobe Acrobat if you use it, and in any printer driver
% that you could use.  You also have to disable the 'scale to fit paper' option
% of your printer driver.
% %
% In any case, please check carefully that the final size of the top and
% bottom margins is 5.2 cm and of the left and right margins is 4.4 cm.
% it is your responsibility to verify this important requirement.  If these margin requirements and not fulfilled at the end of your file generation process, please use the commands at the beginning of the SeminarV2.tex file to correct them.  Otherwise, please do not modify these commands.

% \subsection{Style information}
% Please do not add page numbers to this style; page numbers will be added by the publisher. Do not add headings to your document.
% \subsection{Mathematics}
% You may include additional packages for typesetting
% algorithms, mathematical formula or to define new operators and environments
% if and only if there is no conflict with the SeminarV2.cls
% file.

% It is recommended to avoid the numbering of equations when not
% necessary. When dealing with equation arrays, it could be
% necessary to label several (in)equalities. You can do it using the
% `$\backslash$stackrel' operator (see the SeminarV2.tex source file);
% example:

% \begin{eqnarray}
% c&=&|d|+|e|\nonumber\\
% &\stackrel{\text{(a)}}{=}&d+e\nonumber\\
% &\stackrel{\text{(b)}}{\geq}&\sqrt{f}\enspace,
% \end{eqnarray}
% \noindent where the equality (a) results from the fact that both
% $d$ and $e$ are positive while (b) comes from the definition of
% $f$.

% \subsection{Tables and figures}

% Figure \ref{Fig:MV} shows an example of figure and related
% caption.  Do not use too small symbols and lettering in your
% figures.  Warning: your paper will be printed in black and white
% in the proceedings.  You may insert color figures, but it is your
% responsibility to check that they print correctly in black and
% white.  The color version will be kept in the Seminar electronic
% proceedings available on the web.

% \begin{figure}[ht]
% \centering
% % \includegraphics[scale=0.5]{any_question.eps}
% \mycaption{Any questions?\label{Fig:MV}}
% \end{figure}

% Table \ref{Tab:AgeWeight} shows an example of table.

% \begin{table}[h!]
%   \centering
%   \begin{tabular}{|c|c|c|}
%     \hline
%     ID & age & weight \\
%     \hline
%     1& 15 & 65 \\
%     2& 24 & 74\\
%     3& 18 & 69 \\
%     4& 32 & 78 \\
%     \hline
%   \end{tabular}
%   \mycaption{Age and weight of people.\label{Tab:AgeWeight}}
% \end{table}

% ****************************************************************************
% BIBLIOGRAPHY AREA
% ****************************************************************************

\begin{footnotesize}
\bibliographystyle{unsrt}
\bibliography{own.bib}
\end{footnotesize}

% ****************************************************************************
% END OF BIBLIOGRAPHY AREA
% ****************************************************************************

\end{document}
