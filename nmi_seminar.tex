\documentclass{SeminarV2}
\usepackage[dvips]{graphicx}
\usepackage[utf8]{inputenc}
\usepackage{amssymb,amsmath,array}

%***********************************************************************
% !!!! IMPORTANT NOTICE ON TEXT MARGINS !!!!!
%***********************************************************************
%
% Please avoid using DVI2PDF or PS2PDF converters: some undesired
% shifting/scaling may occur when using these programs
% It is strongly recommended to use the DVIPS converters.
%
% Check that you have set the paper size to A4 (and NOT to letter) in your
% dvi2ps converter, in Adobe Acrobat if you use it, and in any printer driver
% that you could use.  You also have to disable the 'scale to fit paper' option
% of your printer driver.
%
% In any case, please check carefully that the final size of the top and
% bottom margins is 5.2 cm and of the left and right margins is 4.4 cm.
% It is your responsibility to verify this important requirement.  If these margin requirements and not fulfilled at the end of your file generation process, please use the following commands to correct them.  Otherwise, please do not modify these commands.
%
\voffset 0 cm \hoffset 0 cm \addtolength{\textwidth}{0cm}
\addtolength{\textheight}{0cm}\addtolength{\leftmargin}{0cm}

%***********************************************************************
% !!!! USE OF THE SeminarV2 LaTeX STYLE FILE !!!!!
%***********************************************************************
%
% Some commands are inserted in the following .tex example file.  Therefore to
% set up your Seminar submission, please use this file and modify it to insert
% your text, rather than staring from a blank .tex file.  In this way, you will
% have the commands inserted in the right place.

% Edited by Martin Bogdan.

\begin{document}
\title{Seminar Bionanaloge Systeme - Sonja Klein}
\maketitle

\begin{abstract}
  \cite{InstituteofElectricalandElectronicsEngineers.2024}
Inhalt: Fassen Sie kurz das Problem fester Prothesensteifigkeit zusammen. Nennen Sie das Ziel Ihrer Arbeit (Zusammenfassung aktueller Forschung zur bevorzugten Sprunggelenksteifigkeit). Stellen Sie die wichtigsten, aus den Quellen abgeleiteten Ergebnisse dar: z.B. dass die bevorzugte Steifigkeit je nach Aktivität variiert, dass Unterschiede zwischen Patient und Prothetist in der Präferenz bestehen, und dass biomechanische Effekte von der Steifigkeit abhängen. Formulieren Sie die zentrale Schlussfolgerung, dass verstellbare Prothesen notwendig sein könnten, um die Anforderungen verschiedener Aktivitäten optimal zu erfüllen und die Patientenzufriedenheit zu erhöhen.

% ich möchte eine ca 4-6 seiten lange wissenschaftliche arbeit (zusammenfassend) über preferred ankle stiffness in lower limb prothesis schreiben (auf deutsch) und ich habe als Hauptreferenz die arbeit von nicolas j. pett genommen und möchte in die arbeit aber auch die anderen hier hochgeladenen paper nehmen. bitte helfe mir dabei ein sehr gutes paper zu schreiben, indem du mir für jedes kapitel einen kleinen text schreibst was und welche quellen da so ca drin stehen sollten und die gliederung darfst du dir selber ausdenken. ich möchte jedoch auf jeden fall ein abstract, das ca 100 wörter lang ist und ein schluss.

% Type your 100 words abstract here. Please do not modify the style
% of the paper. In particular, keep the text offsets to zero when
% possible (see above in this `SeminarV2.tex' sample file). You may
% \emph{slightly} modify it when necessary, but strictly respecting
% the margin requirements is mandatory (see the instructions to
% authors for more details).
\end{abstract}

\section{Einleitung}
Hintergrund: Herausforderungen bei Amputierten, Mängel aktueller Prothesen *   Aufkommende variable Prothesen als Lösungsansatz *   Forschungslücke: Präferierte Steifigkeit über verschiedene Aktivitäten hinweg *   Ziel der vorliegenden Studie

Inhalt: Beginnen Sie mit der Bedeutung von Unterschenkelprothesen für Menschen nach einer Amputation und den Herausforderungen, denen sie im Alltag begegnen (z.B. ineffizientes Gangbild, eingeschränkte Mobilität, Beschwerden). Erläutern Sie die Rolle der mechanischen Eigenschaften von Prothesenfüßen, insbesondere der Sprunggelenksteifigkeit, für die Gangqualität und die Leistung des Nutzers. Stellen Sie das Problem dar, dass die meisten handelsüblichen Prothesenfüße eine feste Steifigkeit haben, obwohl die biomechanischen Anforderungen je nach Aktivität (Gehen auf unterschiedlichen Oberflächen, Steigungen, Treppen) variieren. Formulieren Sie klar das Ziel Ihrer Arbeit: Aktuelle Forschungsergebnisse zur bevorzugten Sprunggelenksteifigkeit bei Unterschenkelprothesen zusammenzufassen und deren Implikationen für Design und Anpassung von Prothesen, insbesondere verstellbaren Systemen, aufzuzeigen.

\section{Hintergrund und Bedeutung der Sprunggelenksteifigkeit}  
Einschränkungen herkömmlicher Prothesen (Carbon-Verbundfedern mit fester Mechanik) *   Folgen für Nutzer (Unbehagen, kompensatorische Bewegungen, geringere Mobilität, sekundäre Erkrankungen) *   Notwendigkeit der Anpassung der Mechanik an unterschiedliche Aktivitäten *   Konzept variabler Steifigkeit bei Prothesen *   Benutzerpräferenz als potenzielles "Meta-Kriterium" zur intelligenten Anpassung *   Hinweise auf unterschiedliche präferierte Steifigkeit und kinematische/metabolische Vorteile durch Variation

Inhalt: Beschreiben Sie kurz die Funktion eines Prothesenfußes während des Gangzyklus und wie er versucht, die Rolle des biologischen Fußes und Sprunggelenks zu ersetzen (Stoßabsorption, Unterstützung im Stand, Abstoß). Erklären Sie, dass die Sprunggelenksteifigkeit ein zentraler mechanischer Parameter ist, der dieses Verhalten beeinflusst. Gehen Sie darauf ein, dass eine ungeeignete Steifigkeit zu Gangabweichungen führen kann (z.B. Fusschlag, Knie-Hyperextension, verminderte Energiespeicherung), was langfristig zu sekundären Beschwerden (z.B. Schmerzen, Gelenkdegeneration) führen kann. Erwähnen Sie, dass die Suche nach biomechanischen Markern für die "optimale" Steifigkeit bisher nicht eindeutig war, was die Bedeutung der Nutzerwahrnehmung und Präferenz hervorhebt. Stellen Sie die Idee verstellbarer Prothesen als potenziellen Weg vor, um die Steifigkeit an Nutzerbedürfnisse und Aktivitäten anzupassen.

\section{Forschungsansätze und wichtige Ergebnisse}
VSPA Foot (Variable-Stiffness Prosthetic Ankle-Foot) *   Beschreibung und Funktionsweise (Blattfeder, Nockengetriebe, verstellbare Auflage/Schieber) *   Steifigkeitsbereich (200 bis 1060 Nm/rad mit linearer Drehmoment-Winkel-Kurve) *   Verhältnis Plantarflexions- zu Dorsiflexionssteifigkeit (1:3) *   Stellen der Steifigkeit während der Schwungphase *   Vorteile für Nutzerstudien (nahtloses Erleben verschiedener Steifigkeiten, schnelles Finden von Präferenzen, leichtgewichtig) *   Teilnehmer (Anzahl, Profil, Einschlusskriterien) *   Akklimatisierung an die Laborumgebung und die Prothese *   Verfahren zur Bestimmung der präferierten Steifigkeit *   Bedienung über ein elektronisches Drehrad *   Durchführung über fünf Aktivitäten (Ebene, Neigung aufwärts/abwärts, Treppe aufwärts/abwärts) *   Feste Reihenfolge der Aktivitäten aus praktischen Gründen *   Selbstausgewählte Gehgeschwindigkeit auf dem Laufband *   Eigenes Tempo auf der Treppe *   Sieben Präferenz-Identifikationsversuche pro Aktivität *   Startsteifigkeit und Re-Seeding *   Exploration des vollen Steifigkeitsbereichs *   Beschreibung der Lokomotionsaktivitäten *   Ebenerdiges Gehen (Laufband) *   Gehen auf Neigung/Gefälle (Laufband, 4,7º) *   Treppe aufwärts/abwärts (statisches Treppenhaus, Beschreibung der Stufenmaße) *   Nutzung von Handläufen *   Messtechnik *   IMU zur Gangerkennungsphasenschätzung *   Motor zur Steifigkeitsanpassung *   Winkelencoder für Drehradposition und Knöchelkinematik *   Onboard-Computer und Elektronik *   Datenerfassung und -verarbeitung
 
Präferierte Steifigkeit *   Unterschiede der durchschnittlichen präferierten Steifigkeit zwischen den Aktivitäten *   Quantitative Ergebnisse (Nm/rad) für jede Aktivität *   Präferenzen relativ zum ebenen Gehen (\%) *   Unterschiede übersteigen die Wahrnehmungsschwelle (7,7\%) *   Große Variation der Präferenzen zwischen den Teilnehmern innerhalb einer Aktivität *   Geringere Variation innerhalb der Teilnehmer für eine bestimmte Aktivität *   Konsistenz der Präferenzauswahl variiert mit der Aktivität *   Kinematik und Gehgeschwindigkeit *   Unterschiede in der Knöchelkinematik zwischen den Aktivitäten *   Mittlere Spitzendorsalextension für jede Aktivität *   Inverse Beziehung zwischen Prothesensteifigkeit und Bewegungsbereich des Knöchels *   Selbstausgewählte Gehgeschwindigkeiten für Laufbandaktivitäten

Inhalt: Dieses Kapitel ist das Herzstück Ihrer Arbeit, in dem Sie die spezifischen Studien zusammenfassen. Teilen Sie es thematisch auf, um die verschiedenen Aspekte der Forschung zur Steifigkeit darzustellen.
◦
Systematische Untersuchung fester Steifigkeiten (Major et al.): Beschreiben Sie den Ansatz von Major et al., die Effekte fester (niedriger und hoher) Dorsalextension- und Plantarflexionssteifigkeiten auf Gangkinematik, Prothesenbelastung und metabolische Kosten systematisch mittels eines experimentellen Prothesenfußes (CFAM) untersuchten. Fassen Sie die wichtigsten Ergebnisse zusammen: Niedrige Dorsalextensionsteifigkeit führte im Allgemeinen zu größerer Dorsalextensionsbewegung der Prothesenseite, größerer Kniebeugung auf der gesunden Seite, reduzierter Bodenreaktionskraft während der Belastungsphase und reduzierten metabolischen Kosten. Weisen Sie darauf hin, dass die Unterschiede bei Plantarflexionssteifigkeit gering waren und dass die beobachteten Unterschiede, obwohl tendenziell vorteilhaft bei niedriger Dorsalextension, oft klein waren. Erwähnen Sie, dass niedrigere Dorsalextension die tibiale Progression im späten Stand erleichtern könnte.

Quellen:
Taskabhängige biomechanische Effekte verstellbarer Steifigkeit (Ármannsdóttir et al.): Stellen Sie die Studie von Ármannsdóttir et al. vor, die einen neuartigen Prothesenfuß mit verstellbarer Steifigkeit (VSA Fuß) verwendete, um die Effekte auf die Biomechanik (insbesondere Sprunggelenk RoM und Dynamic Joint Stiffness) bei verschiedenen Gangaufgaben (Level, Steigung, Gefälle, verschiedene Geschwindigkeiten) zu untersuchen. Beschreiben Sie, dass die Dorsalextensionswinkel mit weicherem Fuß und höherer Geschwindigkeit/stärkerer Steigung zunahmen. Heben Sie hervor, dass die Effekte der Steifigkeit auf die Prothesendynamik aufgabenabhängig sind, insbesondere bei kinetischen Parametern. Die Ergebnisse deuten darauf hin, dass ein Fuß, dessen Steifigkeit vom Nutzer an die Aufgabe angepasst werden kann, für aktive Personen vorteilhaft sein könnte.

Quellen:
Präferenzunterschiede zwischen Nutzern und Prothetisten (Shepherd and Rouse): Fassen Sie die Studie von Shepherd and Rouse zusammen, die die Präferenzen für Sprunggelenksteifigkeit bei Nutzern und Prothetisten verglich. Das Hauptresultat hierbei war, dass Prothetisten im Durchschnitt eine signifikant höhere Steifigkeit bevorzugten (um 26\%) als die Patienten. Wichtig ist auch die Erkenntnis, dass Patienten deutlich konsistenter in ihrer Präferenz waren als Prothetisten. Diskutieren Sie kurz mögliche Gründe für diese Diskrepanz (z.B. Prothetisten verlassen sich auf visuelle Hinweise wie "Fusschlag" oder "Drop-off", die eher bei niedriger Steifigkeit auffallen, während Patienten das Gefühl der Belastung oder des Energie-Returns stärker wahrnehmen könnten). Erwähnen Sie, dass bei der gemeinsamen Festlegung einer Steifigkeit (Patient und Prothetist kommunizieren) kein eindeutiger Trend erkennbar war, wer den Prozess stärker beeinflusste.
Quellen:
Aufgabenabhängige Präferenz des Nutzers (Pett et al.): Präsentieren Sie ausführlich die Ergebnisse von Pett et al., da dies Ihre Hauptreferenz ist. Beschreiben Sie, dass diese Studie die bevorzugte Sprunggelenksteifigkeit von transtibial amputierten Nutzern über fünf verschiedene Aktivitäten quantifizierte: Level Walking, Steigung, Gefälle, Treppe aufwärts, Treppe abwärts. Das Schlüsselergebnis ist, dass die bevorzugte Steifigkeit erheblich zwischen den Aktivitäten variierte. Geben Sie Beispiele für die Unterschiede an, z.B. dass die Präferenzen zwischen Gefälle-Gehen und Treppe abwärts um 31,8\% der Präferenz für Level Walking differierten. Betonen Sie, dass diese Unterschiede größer waren als die wahrnehmbare Schwelle (Just-Noticeable Difference) und mehreren "Kategorien" kommerzieller Prothesenfüße entsprachen. Stellen Sie die Beziehung zwischen Steifigkeit und Kinematik dar, z.B. dass die Sprunggelenk-Bewegungsamplitude (RoM) umgekehrt zur Steifigkeit war.

\section{Diskussion}
Implikationen der Ergebnisse *   Die Variation der präferierten Steifigkeit unterstreicht die Notwendigkeit variabler Prothesen *   Die festgestellten Unterschiede entsprechen mehreren "Kategorien" kommerzieller Prothesen *   Inner-individuelle Unterschiede in den Präferenzen sind noch größer *   Konventionelle ESR-Prothesen können diesen Bedarf nicht decken *   Die Ergebnisse stimmen mit früheren Studien zum ebenen Gehen überein *   Veränderungen der Steifigkeit beeinflussen die Kinematik und können die Mobilität verbessern *   Zusammenhang zwischen Steifigkeit und Spitzendorsalextension (außer bei Treppen) *   Biomechanische Anforderungen bei Treppen können anders sein *   Einschränkungen der Studie *   Spezifische Eigenschaften des VSPA Foot (flache Unterseite, fehlende Nachgiebigkeit in der Ferse, kein Schuh/Kosmetik) *   Fehlende Längenanpassung an den intakten Fuss, geringes Spiel im Gelenk *   Teilnehmer hielten sich an Handläufen fest *   Nur eine relativ milde Steigung/ein mildes Gefälle untersucht *   Feste Reihenfolge der Aktivitäten *   Begrenzte Beobachtungszeit und mögliche Auswirkungen einer längeren Akklimatisierung
6. Schlussfolgerung *   Bestätigung, dass die präferierte Prothesensteifigkeit bei verschiedenen Aktivitäten deutlich variiert *   Variationen spiegeln Kinematik, Komfort und Balance wider *   Deutlicher Bedarf an variabler Steifigkeit *   Notwendigkeit der Individualisierung der Prothesen *   Ausblick auf zukünftige Forschung (biomechanische Analysen der zugrundeliegenden Faktoren)

◦
Integration der Befunde: Diskutieren Sie, wie die beobachteten biomechanischen Effekte unterschiedlicher Steifigkeiten (Major et al., Ármannsdóttir et al.) die Präferenzmuster der Nutzer (Pett et al.) erklären könnten. Unterschiedliche Aufgaben erfordern unterschiedliche Bewegungen und Kräfte, was durch angepasste Steifigkeiten besser unterstützt werden kann (z.B. höhere RoM bei Steigung erfordert eventuell weichere Steifigkeit).
◦
Implikationen für verstellbare Prothesen: Argumentieren Sie stark auf Basis der Ergebnisse von Pett et al., dass die signifikante Variabilität der bevorzugten Steifigkeit über Aktivitäten hinweg die Notwendigkeit für Prothesen mit anpassbarer oder variabler Steifigkeit unterstreicht. Eine feste Steifigkeit kann nicht optimal für alle Aktivitäten sein. Variable Steifigkeit könnte es Nutzern ermöglichen, die Prothese an die Anforderungen der spezifischen Aufgabe anzupassen, was potenziell Gangqualität, Komfort und Leistung verbessert.
◦
Patientenpräferenz in der klinischen Anpassung: Diskutieren Sie die Befunde von Shepherd and Rouse im Kontext der klinischen Praxis. Die Diskrepanz in der Präferenz zwischen Prothetisten und Patienten und die höhere Konsistenz der Patienten legen nahe, dass die Rückmeldung des Patienten bei der Anpassung der Steifigkeit eine entscheidende, vielleicht unterschätzte Rolle spielen sollte. Variable Steifigkeit könnte den Anpassungsprozess erleichtern, indem sie Patient und Prothetist ermöglicht, verschiedene Einstellungen effizient zu erkunden.
◦
Einschränkungen der Studien: Reflektieren Sie kritisch die Limitationen der vorgestellten Studien, wie sie von den Autoren genannt werden: kleine Stichprobengrößen, die Verwendung experimenteller Prothesen, deren Eigenschaften von kommerziellen Füßen abweichen können, der Einfluss der Prothesenausrichtung, die in einigen Studien konstant gehalten wurde, die kurze Anpassungszeit an neue Einstellungen oder Geräte, und die spezifischen getesteten Aktivitäten/Steigungen.
◦
Ausblick auf zukünftige Forschung: Leiten Sie aus den Einschränkungen und offenen Fragen die Notwendigkeit weiterer Forschung ab. Vorschläge könnten sein: Studien mit größeren, vielfältigeren Stichproben; Untersuchung eines breiteren Spektrums von Aktivitäten, Geschwindigkeiten und Steifigkeitsbereichen; Langzeitstudien zur Anpassung; die Entwicklung von Methoden zur Identifizierung der wirklich optimalen (nicht nur bevorzugten) Steifigkeit; Untersuchung des Einflusses der Ausrichtung im Zusammenspiel mit variabler Steifigkeit; und die Erprobung nutzersteuerbarer verstellbarer Systeme im Alltag.

\section{Schlussfolgerung}
Inhalt: Fassen Sie die wichtigsten Erkenntnisse prägnant zusammen. Wiederholen Sie, dass die Forschung deutlich zeigt, dass eine feste Sprunggelenksteifigkeit in Prothesen die optimale Leistung und den Komfort über das Spektrum der alltäglichen Aktivitäten hinweg einschränkt. Die bevorzugte Steifigkeit variiert signifikant je nach Aufgabe, und die Präferenzen von Nutzern und Prothetisten können unterschiedlich sein. Das Verständnis und die Berücksichtigung der Nutzerpräferenz ist entscheidend für eine erfolgreiche Anpassung. Betonen Sie, dass die Entwicklung und Implementierung von Prothesen mit variabler oder anpassbarer Steifigkeit das Potenzial hat, die Mobilität, den Komfort und letztlich die Lebensqualität von Menschen mit Unterschenkelamputation erheblich zu verbessern. Schließen Sie mit einem Ausblick auf die fortlaufende Relevanz und die Notwendigkeit zukünftiger Forschung in diesem Bereich.

\section{Typesetting an Seminar document using \LaTeX}

xx

\subsection{Page format and margins}
Please avoid using DVI2PDF or PS2PDF converters: some undesired
shifting/scaling may occur when using these programs
It is strongly recommended to use the DVIPS converters. 
%
Check that you have set the paper size to A4 (and NOT to letter) in your
dvi2ps converter, in Adobe Acrobat if you use it, and in any printer driver
that you could use.  You also have to disable the 'scale to fit paper' option
of your printer driver.
%
In any case, please check carefully that the final size of the top and
bottom margins is 5.2 cm and of the left and right margins is 4.4 cm.
it is your responsibility to verify this important requirement.  If these margin requirements and not fulfilled at the end of your file generation process, please use the commands at the beginning of the SeminarV2.tex file to correct them.  Otherwise, please do not modify these commands.

\subsection{Style information}
Please do not add page numbers to this style; page numbers will be added by the publisher. Do not add headings to your document.
\subsection{Mathematics}
You may include additional packages for typesetting
algorithms, mathematical formula or to define new operators and environments
if and only if there is no conflict with the SeminarV2.cls
file.

It is recommended to avoid the numbering of equations when not
necessary. When dealing with equation arrays, it could be
necessary to label several (in)equalities. You can do it using the
`$\backslash$stackrel' operator (see the SeminarV2.tex source file);
example:

\begin{eqnarray}
c&=&|d|+|e|\nonumber\\
&\stackrel{\text{(a)}}{=}&d+e\nonumber\\
&\stackrel{\text{(b)}}{\geq}&\sqrt{f}\enspace,
\end{eqnarray}
\noindent where the equality (a) results from the fact that both
$d$ and $e$ are positive while (b) comes from the definition of
$f$.

\subsection{Tables and figures}

Figure \ref{Fig:MV} shows an example of figure and related
caption.  Do not use too small symbols and lettering in your
figures.  Warning: your paper will be printed in black and white
in the proceedings.  You may insert color figures, but it is your
responsibility to check that they print correctly in black and
white.  The color version will be kept in the Seminar electronic
proceedings available on the web.

\begin{figure}[ht]
\centering
% \includegraphics[scale=0.5]{any_question.eps}
\mycaption{Any questions?\label{Fig:MV}}
\end{figure}

Table \ref{Tab:AgeWeight} shows an example of table.

\begin{table}[h!]
  \centering
  \begin{tabular}{|c|c|c|}
    \hline
    ID & age & weight \\
    \hline
    1& 15 & 65 \\
    2& 24 & 74\\
    3& 18 & 69 \\
    4& 32 & 78 \\
    \hline
  \end{tabular}
  \mycaption{Age and weight of people.\label{Tab:AgeWeight}}
\end{table}

% ****************************************************************************
% BIBLIOGRAPHY AREA
% ****************************************************************************

\begin{footnotesize}
\bibliographystyle{unsrt}
\bibliography{own.bib}
\end{footnotesize}

% ****************************************************************************
% END OF BIBLIOGRAPHY AREA
% ****************************************************************************

\end{document}
