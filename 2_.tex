\section{Einleitung}
Nach einer Unterschenkelamputation soll der prothetische Fuß das fehlende Sprunggelenk ersetzen \cite{Louessard.2022}. Es ist wichtig, dass diese Prothese möglichst gut die Funktionalität eines anatomischen Fußes ersetzt \cite{Stevens.2018}, denn Personen mit Unterschenkelamputationen haben ein energieineffizienten Gang, eine eingeschränkte Mobilität und eine eingeschränkte Gangstabilität \cite{Major.2014}\cite{Vaca.2022}.
Die Prothese muss an die individuellen Bedürfnisse des Nutzers angepasst werden \cite{Stevens.2018}. Dabei spielt die Steifigkeit einer Prothese die Schlüsselrolle \cite{Shepherd.2020}. 
Sie variiert je nach Modell und Kategorie der Prothese und ist wichtig für das Gesamtverhalten des Prothesenfußes, denn sie bestimmt, wie Energie beim Aufprall auf den Boden absorbiert und zurückgegeben wird und wie der Fuß in der terminalen Standphase des Ganges Halt bietet \cite{Shepherd.2020}. Die individuell richtige Steifigkeit für einen Patienten ist entscheidend für seine Gesamtmobilität und möglicherweise langfristige Gesundheit \cite{Shepherd.2020}. 
Ein zu steifer Fuß kann zu einer erhöhten Stoßbelastung und einem verlängerten Fersenkontakt während des frühen Stands, einer Überstreckung des Knies während des terminalen Stands und einer verringerten Speicherung und Rückgabe von Energie führen \cite{Shepherd.2020}. Ein zu nachgiebiger Fuß kann zu einem auffälligen, hörbaren Fußaufschlag in der frühen Gangphase und dem Verlust der vorderen Stütze in der Gangendphase führen \cite{Shepherd.2020}. 
Ganganomalien und Asymmetrien können zu langfristigen Nebenwirkungen wie chronische Rückenschmerzen oder Osteoarthritis führen \cite{Shepherd.2020}.

% muss hier eine Zusammenfassung ud ein Ziel rein?

\section{Verschiedene Prothesentypen für Patienten mit Unterschenkelamputationen}  % hier noch die Überschrift aussagekräftiger machen#
Prothesen gibt es in unterschiedlichen Preisklassen und vielfältigen Designs \cite{Stevens.2018}.
Einfache Solid-ankle-cushion-heel (SACH)-Füße bestehen aus einem massiven Knöchelblock und einem kompressiven Material in der Ferse \cite{Stevens.2018}. Der single-axis-Fuß hat besitzt ein mechanisches Gelenk, um ein Sprunggelenk nachzubilden, der multi-axis-Fuß hat flexible Elemente und ermöglicht so eine gedämpfte Bewegung in allen Bewegungsebenen und mit dem flexible-keel-Fuß wird die Standphase noch durch flexible Elemente im Vorfuß verbessert \cite{Stevens.2018}. 
Außerdem gibt es Energy-storing-and-returning (ESAR)-Füße aus elastischen Materialien, die sich unter Belastung verformen und anschließend wieder in ihre Ursprungsform zurückkehren \cite{Stevens.2018}. Die während der Verformung gespeicherte Energie wird am Ende freigesetzt, um den Gangzyklus mit Energie zu versorgen \cite{Stevens.2018}. 
Der ESAR-Fuß kann die nichtlineare Form der Drehmoment-Winkel-Kurve während der Standphase des Gangs jedoch nicht angemessen nachahmen \cite{Shepherd.2017}. Außerdem ist seine Steifigkeit für das Gehen in der Ebene optimiert, nicht auf beispielweise das Auf- und Absteigen von Treppen oder Rampen oder das Gehen auf unebenem Gelände \cite{Shepherd.2017}. Diese Aktivitäten benötigen unterschiedliche Steifigkeiten \cite{Shepherd.2017}. Um die Steifigkeit anpassen zu können, gibt es quasi-passive Prothesen \cite{Shepherd.2017}.

% muss oder will ich hier einen Absatz?
Der Variable Stiffness Prosthetic Ankle (VSPA)-Fuß ist eine quasi-passive Prothese mit einer nichtlinearen, individuell anpassbaren Drehmoment-Winkel-Kurve, die ihre Gesamtsteifigkeit während der Nutzung in Echtzeit anpassen kann \cite{Shepherd.2017}. Sie kann die biomechanischen Eigenschaften des menschlichen Sprunggelenks genauer nachbilden \cite{Shepherd.2017}. Der Fuß nutzt eine Blattfeder variabler Länge, ein nockenbasiertes Getriebe \cite{Shetty.2022}. Ein kleiner, batteriebetriebener Motor kann die Blattfeder und somit die Steifigkeit während der Schwungphase des Ganges anpassen \cite{Shetty.2022}. % hier noch mehr schreiben (shepherd 2017?) über motorisierte Prothesen?
Es gibt außerdem Forschung zu motorisierten Prothesen \cite{Shetty.2022}. Diese versuchen die Defizite herkömmlicher Prothesen zu vermeiden, sind aber teuer, schwer und wenig robust \cite{Shetty.2022}. Quasi-passive Prothesen mit einem kleinen Motor und einer Batterie versuchen dagegen einen Kompromiss zwischen der Funktionalität komplexer Systeme und deren Nachteilen zu finden \cite{Shetty.2022}. 

% zu viele Wörter in bold?

% Erläutern Sie die Rolle der mechanischen Eigenschaften von Prothesenfüßen, insbesondere der Sprunggelenksteifigkeit, für die Gangqualität und die Leistung des Nutzers. Stellen Sie das Problem dar, dass die meisten handelsüblichen Prothesenfüße eine feste Steifigkeit haben, obwohl die biomechanischen Anforderungen je nach Aktivität (Gehen auf unterschiedlichen Oberflächen, Steigungen, Treppen) variieren. Formulieren Sie klar das Ziel Ihrer Arbeit: Aktuelle Forschungsergebnisse zur bevorzugten Sprunggelenksteifigkeit bei Unterschenkelprothesen zusammenzufassen und deren Implikationen für Design und Anpassung von Prothesen, insbesondere verstellbaren Systemen, aufzuzeigen.
