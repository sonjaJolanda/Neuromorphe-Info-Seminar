\section{Einleitung}
Hintergrund: Herausforderungen bei Amputierten, Mängel aktueller Prothesen *   Aufkommende variable Prothesen als Lösungsansatz *   Forschungslücke: Präferierte Steifigkeit über verschiedene Aktivitäten hinweg *   Ziel der vorliegenden Studie

Inhalt: Beginnen Sie mit der Bedeutung von Unterschenkelprothesen für Menschen nach einer Amputation und den Herausforderungen, denen sie im Alltag begegnen (z.B. ineffizientes Gangbild, eingeschränkte Mobilität, Beschwerden). Erläutern Sie die Rolle der mechanischen Eigenschaften von Prothesenfüßen, insbesondere der Sprunggelenksteifigkeit, für die Gangqualität und die Leistung des Nutzers. Stellen Sie das Problem dar, dass die meisten handelsüblichen Prothesenfüße eine feste Steifigkeit haben, obwohl die biomechanischen Anforderungen je nach Aktivität (Gehen auf unterschiedlichen Oberflächen, Steigungen, Treppen) variieren. Formulieren Sie klar das Ziel Ihrer Arbeit: Aktuelle Forschungsergebnisse zur bevorzugten Sprunggelenksteifigkeit bei Unterschenkelprothesen zusammenzufassen und deren Implikationen für Design und Anpassung von Prothesen, insbesondere verstellbaren Systemen, aufzuzeigen.

\section{Hintergrund und Relevanz der Forschung}  % hier noch die Überschrift aussagekräftiger machen
Bei Personen mit Amputationen lässt sich ein energieineffizienter Gang und eine eingeschränkte Mobilität feststellen \cite{Major.2014}.
Es ist als wichtig, dass eine Prothese möglichst gut die Funktionalität eines anatomischen Fußes ersetzt \cite{Stevens.2018}. Dabei gibt es Prothesen in jeder Preisklasse und vielfältigem Design \cite{Stevens.2018}.

Einfache \textbf{Solid-ankle-cushion-heel (SACH)-Füße} bestehen aus einem massiven Knöchelblock und einem kompressiven Material in der Ferse \cite{Stevens.2018}. Der \textbf{single-axis-Fuß} hat besitzt ein mechanisches Gelenk, um ein Sprunggelenk nachzubilden, der \textbf{multi-axis-Fuß} hat flexible Elemente und ermöglicht so eine gedämpfte Bewegung in allen Bewegungsebenen und mit dem \textbf{flexible-keel-Fuß} wird die Standphase noch durch flexible Elemente im Vorfuß verbessert \cite{Stevens.2018}. 
Außerdem gibt es \textbf{Energy-storing-and-returning (ESAR)-Füße} aus elastischen Materialien, die sich unter Belastung verformen und anschließend wieder in ihre Ursprungsform zurückkehren, wobei die während der Verformung gespeicherte Energie am Ende freigesetzt wird, um den Gangzyklus mit Energie zu versorgen \cite{Stevens.2018}.

Bei dieser Vielfalt ist es wichtig, die Prothese an die Bedürfnisse des Nutzers anzupassen \cite{Stevens.2018}. Dabei spielt die Steifigkeit einer Prothese die Schlüsselrolle \cite{Shepherd.2020}. 
Sie variiert je nach Modell und Kategorie der Prothese und ist wichtig für das Gesamtverhalten des Prothesenfußes, denn sie bestimmt, wie Energie beim Aufprall auf den Boden absorbiert und zurückgegeben wird und wie der Fuß in der terminalen Standphase des Ganges Halt bietet \cite{Shepherd.2020}. Die individuell richtige Steifigkeit für einen Patienten ist entscheidend für seine Gesamtmobilität und möglicherweise langfristige Gesundheit \cite{Shepherd.2020}. 
Ein zu steifer Fuß kann zu einer erhöhten Stoßbelastung und einem verlängerten Fersenkontakt während des frühen Stands, einer Überstreckung des Knies während des terminalen Stands und einer verringerten Speicherung und Rückgabe von Energie führen \cite{Shepherd.2020}. 
Ein zu nachgiebiger Fuß kann zu einem Foot Slap in der frühen Gangphase und dem Verlust der vorderen Stütze in der Gangendphase führen \cite{Shepherd.2020}. 
Ganganomalien und Asymmetrien können zu langfristigen Nebenwirkungen wie chronische Rückenschmerzen oder Osteopenie führen \cite{Shepherd.2020}. \\

\section{Forschung zur Steifigkeit von Prothesenfüßen} 
Es ist also wichtig das Design von Prothesen weiter zu erforschen \cite{Major.2014}. \textcolor{red}{XX} untersuchten 2014 die Auswirkungen unterschiedlicher Rotationssteifigkeiten im Sprunggelenk einer experimentellen Fuß-Knöchel-Gelenkprothese auf verschiedene Gehparameter \cite{Major.2014}.  % TODO 
Es wurde eine individuell angefertigte experimentelle Prothese, eine CFAM, verwendet, die eine systematische Variation mechanischer Eigenschaften ermöglichte, während andere Eigenschaften konstant bleiben konnten \cite{Major.2014}. 
Die Ergebnisse der Studie deuten darauf hin, dass die Gehleistung bei Prothesen mit geringerer Dorsalextension (Hebung des Fußes in Richtung Schienbein) Steifigkeit profitieren könnte \cite{Major.2014}. % kann ich das in Klammern lassen?
Die geringere Dorsalextension-Steifigkeit führte zu einer größeren maximalen Dorsalextension des Sprunggelenks und zu einer größeren maximalen Beugung des Kniegelenks der gesunden Seite im Stehen \cite{Major.2014}. Außerdem führte sie tendenziell zu einer reduzierten maximalen Bodenreaktionskraft während des Gehens, was potenziell vorteilhaft für die Gelenke und das Restglied sein könnte \cite{Major.2014}.  Zudem führte die geringere Dorsalextension-Steifigkeit zu einem geringeren Energierverbrauch beim Gehen, möglicherweise jedoch ohne klinische Signifikant \cite{Major.2014}.

Eine weitere Studie von \textcolor{red}{XX} konzentrierte sich auf die Analyse der Wahrnehmungen von Prothesenträgern und Orthopädietechnikern \cite{Shepherd.2020}. Forscher stützen ihre Empfehlungen auf biometrische Analysen, während Orthopdietechniker sich auf qualitatives Feedback von Patienten und die visuelle Beurteilung des Ganges stützen \cite{Shepherd.2020}. Aus diesem Grund ist eine qualitative hochwertige Analyse der Wahrnehmung wichtig \cite{Shepherd.2020}. Patienten sind in der Lage Faktoren wie Komfort der Fußes, Leichtgängigkeit der Bewegung, Vertrauen in das Gleichgewicht oder auch die lokale Muskelermüdung wahrzunehmen \cite{Shepherd.2020}. Orthopädietechniker können sich neben diesem Feedback auch auf ihre visuelle Einschätzung des Ganges des Patienten und jahrelange Erfahrung stützen \cite{Shepherd.2020}. Die Studie betrachtet beide dieser Wahrnehmungen \cite{Shepherd.2020}. %Ergebnisse


\cite{Clites.2021} % dann das Paper auch kurz zusammenfassen

\cite{Louessard.2022} % dann das Paper auch kurz zusammenfassen 

\cite{Vaca.2022} % dann das Paper auch kurz zusammenfassen


\dots \\



Einschränkungen herkömmlicher Prothesen (Carbon-Verbundfedern mit fester Mechanik) *   Folgen für Nutzer (Unbehagen, kompensatorische Bewegungen, geringere Mobilität, sekundäre Erkrankungen) *   Notwendigkeit der Anpassung der Mechanik an unterschiedliche Aktivitäten *   Konzept variabler Steifigkeit bei Prothesen *   Benutzerpräferenz als potenzielles "Meta-Kriterium" zur intelligenten Anpassung *   Hinweise auf unterschiedliche präferierte Steifigkeit und kinematische/metabolische Vorteile durch Variation

Inhalt: 

Beschreiben Sie kurz die Funktion eines Prothesenfußes während des Gangzyklus und wie er versucht, die Rolle des biologischen Fußes und Sprunggelenks zu ersetzen (Stoßabsorption, Unterstützung im Stand, Abstoß). 

% DONE: Erklären Sie, dass die Sprunggelenksteifigkeit ein zentraler mechanischer Parameter ist, der dieses Verhalten beeinflusst. Gehen Sie darauf ein, dass eine ungeeignete Steifigkeit zu Gangabweichungen führen kann (z.B. Fusschlag, Knie-Hyperextension, verminderte Energiespeicherung), was langfristig zu sekundären Beschwerden (z.B. Schmerzen, Gelenkdegeneration) führen kann.

Erwähnen Sie, dass die Suche nach biomechanischen Markern für die "optimale" Steifigkeit bisher nicht eindeutig war, was die Bedeutung der Nutzerwahrnehmung und Präferenz hervorhebt. Stellen Sie die Idee verstellbarer Prothesen als potenziellen Weg vor, um die Steifigkeit an Nutzerbedürfnisse und Aktivitäten anzupassen.