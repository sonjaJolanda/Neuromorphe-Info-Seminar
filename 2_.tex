\section{Einleitung}
Personen mit Amputationen der unteren Extremitäten haben ein energieineffizienten Gang und eine eingeschränkte Mobilität \cite{Major.2014}. Es ist daher wichtig, dass eine Prothese möglichst gut die Funktionalität eines anatomischen Fußes ersetzt \cite{Stevens.2018}. Prothesen haben unterschiedlichen Preisklassen und vielfältige Designs \cite{Stevens.2018}.
% Hintergrund: Herausforderungen bei Amputierten, Mängel aktueller Prothesen *   Aufkommende variable Prothesen als Lösungsansatz *   Forschungslücke: Präferierte Steifigkeit  *   Ziel der vorliegenden Studie

% Inhalt: Beginnen Sie mit der Bedeutung von Unterschenkelprothesen für Menschen nach einer Amputation und den Herausforderungen, denen sie im Alltag begegnen (z.B. ineffizientes Gangbild, eingeschränkte Mobilität, Beschwerden). Erläutern Sie die Rolle der mechanischen Eigenschaften von Prothesenfüßen, insbesondere der Sprunggelenksteifigkeit, für die Gangqualität und die Leistung des Nutzers. Stellen Sie das Problem dar, dass die meisten handelsüblichen Prothesenfüße eine feste Steifigkeit haben, obwohl die biomechanischen Anforderungen je nach Aktivität (Gehen auf unterschiedlichen Oberflächen, Steigungen, Treppen) variieren. Formulieren Sie klar das Ziel Ihrer Arbeit: Aktuelle Forschungsergebnisse zur bevorzugten Sprunggelenksteifigkeit bei Unterschenkelprothesen zusammenzufassen und deren Implikationen für Design und Anpassung von Prothesen, insbesondere verstellbaren Systemen, aufzuzeigen.

\section{Verschiedene Prothesen und Steifigkeit}  % hier noch die Überschrift aussagekräftiger machen#
Einfache \textbf{Solid-ankle-cushion-heel (SACH)-Füße} bestehen aus einem massiven Knöchelblock und einem kompressiven Material in der Ferse \cite{Stevens.2018}. Der \textbf{single-axis-Fuß} hat besitzt ein mechanisches Gelenk, um ein Sprunggelenk nachzubilden, der \textbf{multi-axis-Fuß} hat flexible Elemente und ermöglicht so eine gedämpfte Bewegung in allen Bewegungsebenen und mit dem \textbf{flexible-keel-Fuß} wird die Standphase noch durch flexible Elemente im Vorfuß verbessert \cite{Stevens.2018}. 
Außerdem gibt es \textbf{Energy-storing-and-returning (ESAR)-Füße} aus elastischen Materialien, die sich unter Belastung verformen und anschließend wieder in ihre Ursprungsform zurückkehren, wobei die während der Verformung gespeicherte Energie am Ende freigesetzt wird, um den Gangzyklus mit Energie zu versorgen \cite{Stevens.2018}. % hier noch mal schauen ob es neuere Literatur zur Klassifikation von Prothesenfüßen gibt

Es ist wichtig, die Prothese an die Bedürfnisse des Nutzers anzupassen \cite{Stevens.2018}. Dabei spielt die Steifigkeit einer Prothese die Schlüsselrolle \cite{Shepherd.2020}. 
Sie variiert je nach Modell und Kategorie der Prothese und ist wichtig für das Gesamtverhalten des Prothesenfußes, denn sie bestimmt, wie Energie beim Aufprall auf den Boden absorbiert und zurückgegeben wird und wie der Fuß in der terminalen Standphase des Ganges Halt bietet \cite{Shepherd.2020}. Die individuell richtige Steifigkeit für einen Patienten ist entscheidend für seine Gesamtmobilität und möglicherweise langfristige Gesundheit \cite{Shepherd.2020}. 
Ein zu steifer Fuß kann zu einer erhöhten Stoßbelastung und einem verlängerten Fersenkontakt während des frühen Stands, einer Überstreckung des Knies während des terminalen Stands und einer verringerten Speicherung und Rückgabe von Energie führen \cite{Shepherd.2020}. Ein zu nachgiebiger Fuß kann zu einem auffälligen, hörbaren Fußaufschlag in der frühen Gangphase und dem Verlust der vorderen Stütze in der Gangendphase führen \cite{Shepherd.2020}. 
Ganganomalien und Asymmetrien können zu langfristigen Nebenwirkungen wie chronische Rückenschmerzen oder Osteoarthritis führen \cite{Shepherd.2020}. \\

Der ESAR-Fuß kann die nichtlineare Form der Drehmoment-Winkel-Kurve während der Standphase des Gangs nicht angemessen nachahmen \cite{Shepherd.2017}. Außerdem ist die Steifigkeit dieser Prothese für das Gehen auf ebenem Boden optimiert, nicht auf andere Situationen, wie das Auf- und Absteigen von Treppen oder Rampen oder dem Gehen auf unebenem Gelände \cite{Shepherd.2017}. Diese sehr unterschiedlichen Aktivitäten benötigen unterschiedliche Steifigkeiten \cite{Shepherd.2017}. Um die Steifigkeit anpassen zu können, gibt es quasi-passive Prothesen \cite{Shepherd.2017}. 

Der \textbf{Variable Stiffness Prosthetic Ankle (VSPA)}-Fuß ist eine quasi-passive Prothese mit einer nichtlinearen, individuell anpassbaren Drehmoment-Winkel-Kurve, die ihre Gesamtsteifigkeit während der Nutzung in Echtzeit anpassen kann \cite{Shepherd.2017}. Die Drehmoment-Winkel-Kurve des VSPA-Fußes kann die biomechanischen Eigenschaften des menschlichen Sprunggelenks genauer nachbilden \cite{Shepherd.2017}. Der Fuß nutzt eine Blattfeder \textcolor{red}{(ist das das richtige Wort???)} variabler Länge und ein nockenbasiertes Getriebe \cite{Shetty.2022}. Ein kleiner, batteriebetriebener Motor kann die Blattfeder und somit die Steifigkeit während der Schwungphase des Ganges anpassen \cite{Shetty.2022}. % hier noch mehr schreiben 

%  shepherd 2017 noch durchlesen und vielleicht hier noch Sachen dazu rein schreiben 

% TODO: Beschreiben Sie kurz die Funktion eines Prothesenfußes während des Gangzyklus und wie er versucht, die Rolle des biologischen Fußes und Sprunggelenks zu ersetzen (Stoßabsorption, Unterstützung im Stand, Abstoß). 
% DONE: Erklären Sie, dass die Sprunggelenksteifigkeit ein zentraler mechanischer Parameter ist, der dieses Verhalten beeinflusst. Gehen Sie darauf ein, dass eine ungeeignete Steifigkeit zu Gangabweichungen führen kann (z.B. Fusschlag, Knie-Hyperextension, verminderte Energiespeicherung), was langfristig zu sekundären Beschwerden (z.B. Schmerzen, Gelenkdegeneration) führen kann.