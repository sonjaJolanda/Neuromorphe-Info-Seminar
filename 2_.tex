\section{Einleitung}
Hintergrund: Herausforderungen bei Amputierten, Mängel aktueller Prothesen *   Aufkommende variable Prothesen als Lösungsansatz *   Forschungslücke: Präferierte Steifigkeit über verschiedene Aktivitäten hinweg *   Ziel der vorliegenden Studie

Inhalt: Beginnen Sie mit der Bedeutung von Unterschenkelprothesen für Menschen nach einer Amputation und den Herausforderungen, denen sie im Alltag begegnen (z.B. ineffizientes Gangbild, eingeschränkte Mobilität, Beschwerden). Erläutern Sie die Rolle der mechanischen Eigenschaften von Prothesenfüßen, insbesondere der Sprunggelenksteifigkeit, für die Gangqualität und die Leistung des Nutzers. Stellen Sie das Problem dar, dass die meisten handelsüblichen Prothesenfüße eine feste Steifigkeit haben, obwohl die biomechanischen Anforderungen je nach Aktivität (Gehen auf unterschiedlichen Oberflächen, Steigungen, Treppen) variieren. Formulieren Sie klar das Ziel Ihrer Arbeit: Aktuelle Forschungsergebnisse zur bevorzugten Sprunggelenksteifigkeit bei Unterschenkelprothesen zusammenzufassen und deren Implikationen für Design und Anpassung von Prothesen, insbesondere verstellbaren Systemen, aufzuzeigen.

\section{Hintergrund}  % hier noch die Überschrift aussagekräftiger machen
Ein wichtiger Teil einer Prothese für die unteren Gliedmaßen ist der Prothesenfuß \cite{Stevens.2018}. Aus diesem Grund, gibt es viele verschiedene Konstruktionen in verschiedenen Preisklassen um die Funktionalität eines anatomischen Fußes zu ersetzen \cite{Stevens.2018}. Einige Füße bestehen heute beispielsweise aus elastischen Materialien, die sich unter Belastung verformen und anschließend wieder in ihre Ursprungsform zurückkehren, wobei die während der Verformung gespeicherte Energie am Ende freigesetzt wird, um den Gangzyklus mit Energie zu versorgen \cite{Stevens.2018}. 

Diese große Anzahl an verschiedenen Prothesenfüßen lassen sich in verschiedene Kategorien einteilen \cite{Stevens.2018}. Der \textbf{Solid-ankle-cushion-heel (SACH)-Fuß} ist die einfachste Kategorie \cite{Stevens.2018}. Sie besteht aus einem massiven Knöchelblock und einem kompressiven Material in der Ferse \cite{Stevens.2018}. Der \textbf{single-axis-Fuß} hat besitzt ein mechanisches Gelenk, um ein Sprunggelenk nachzubilden \cite{Stevens.2018}. Der \textbf{multi-axis-Fuß} hat flexible Elemente und ermöglicht so eine gedämpfte Bewegung in allen Bewegungsebenen und mit dem \textbf{flexible-keel-Fuß} wird die Standphase noch durch flexible Elemente im Vorfuß verbessert \cite{Stevens.2018}. Eine weitere Kategorie sind \textbf{Energy-storing-and-returning (ESAR)-Füße}, die wie bereits erwähnt aus elastischen Materialien bestehen \cite{Stevens.2018}.  \\

Bei Personen mit Amputationen lässt sich ein energieineffizienter Gang und eine eingeschränkte Mobilität feststellen \cite{Major.2014}. Es ist also wichtig Prothesen besser zu verstehen umd diese Situation zu verbessern \cite{Major.2014}. 

% Ich glaube ich möchte diese beiden Paper zusammebringen: The effects of prosthetic ankle stiffness on ankle and knee kinematics, prosthetic limb loading, and net metabolic cost of trans-tibial amputee gait

\dots \\



Einschränkungen herkömmlicher Prothesen (Carbon-Verbundfedern mit fester Mechanik) *   Folgen für Nutzer (Unbehagen, kompensatorische Bewegungen, geringere Mobilität, sekundäre Erkrankungen) *   Notwendigkeit der Anpassung der Mechanik an unterschiedliche Aktivitäten *   Konzept variabler Steifigkeit bei Prothesen *   Benutzerpräferenz als potenzielles "Meta-Kriterium" zur intelligenten Anpassung *   Hinweise auf unterschiedliche präferierte Steifigkeit und kinematische/metabolische Vorteile durch Variation

Inhalt: Beschreiben Sie kurz die Funktion eines Prothesenfußes während des Gangzyklus und wie er versucht, die Rolle des biologischen Fußes und Sprunggelenks zu ersetzen (Stoßabsorption, Unterstützung im Stand, Abstoß). Erklären Sie, dass die Sprunggelenksteifigkeit ein zentraler mechanischer Parameter ist, der dieses Verhalten beeinflusst. Gehen Sie darauf ein, dass eine ungeeignete Steifigkeit zu Gangabweichungen führen kann (z.B. Fusschlag, Knie-Hyperextension, verminderte Energiespeicherung), was langfristig zu sekundären Beschwerden (z.B. Schmerzen, Gelenkdegeneration) führen kann. Erwähnen Sie, dass die Suche nach biomechanischen Markern für die "optimale" Steifigkeit bisher nicht eindeutig war, was die Bedeutung der Nutzerwahrnehmung und Präferenz hervorhebt. Stellen Sie die Idee verstellbarer Prothesen als potenziellen Weg vor, um die Steifigkeit an Nutzerbedürfnisse und Aktivitäten anzupassen.