\section{Diskussion und Schlussfolgerung}
Die Ergebnisse der verschiedenen Studien zeigten, dass die richtige Steifigkeit für einzelne Patienten und Aktivitäten wichtig beim Design einer Prothese ist. Die individuelle Steifigkeit ist ein zentraler Faktor für die Mobilität und das Wohlbefinden von Patienten mit Unterschenkelamputation und sie variiert je nach Aktivität. Eine spannender Ansatz ist das Vorhersagen der bevorzugten Steifigkeit mit Hilfe von maschinellen Lernverfahren, mit denen dann Prothesen ihre Steifigkeit während des Ganges anpassen könnten. Es ist ein vielversprechender Ansatz, der allerdings noch viel Potenzial für weitere Forschung bietet. 

% in research rabbit gab es noch drei Paper die dieses Paper zitieren, die könnte ich mir an sich auch noch ansehen 

% put my opinion in the paper
% es hieß doch irgendwie mal man soll nicht in der Vergangenheit schreiben, aber hier macht Vergangenheit ja irgendwie Sinn?
% am Ende noch Synonyme für das Wort Patienten finden 
% am Schluss noch mal mit der Vorlage von Bogdan vergleichen
% auch Rechtschreibung und Grammatikfehler prüfen lassen