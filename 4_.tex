\section{Diskussion}
Implikationen der Ergebnisse *   Die Variation der präferierten Steifigkeit unterstreicht die Notwendigkeit variabler Prothesen *   Die festgestellten Unterschiede entsprechen mehreren "Kategorien" kommerzieller Prothesen *   Inner-individuelle Unterschiede in den Präferenzen sind noch größer *   Konventionelle ESR-Prothesen können diesen Bedarf nicht decken *   Die Ergebnisse stimmen mit früheren Studien zum ebenen Gehen überein *   Veränderungen der Steifigkeit beeinflussen die Kinematik und können die Mobilität verbessern *   Zusammenhang zwischen Steifigkeit und Spitzendorsalextension (außer bei Treppen) *   Biomechanische Anforderungen bei Treppen können anders sein *   Einschränkungen der Studie *   Spezifische Eigenschaften des VSPA Foot (flache Unterseite, fehlende Nachgiebigkeit in der Ferse, kein Schuh/Kosmetik) *   Fehlende Längenanpassung an den intakten Fuss, geringes Spiel im Gelenk *   Teilnehmer hielten sich an Handläufen fest *   Nur eine relativ milde Steigung/ein mildes Gefälle untersucht *   Feste Reihenfolge der Aktivitäten *   Begrenzte Beobachtungszeit und mögliche Auswirkungen einer längeren Akklimatisierung
6. Schlussfolgerung *   Bestätigung, dass die präferierte Prothesensteifigkeit bei verschiedenen Aktivitäten deutlich variiert *   Variationen spiegeln Kinematik, Komfort und Balance wider *   Deutlicher Bedarf an variabler Steifigkeit *   Notwendigkeit der Individualisierung der Prothesen *   Ausblick auf zukünftige Forschung (biomechanische Analysen der zugrundeliegenden Faktoren)

◦
Integration der Befunde: Diskutieren Sie, wie die beobachteten biomechanischen Effekte unterschiedlicher Steifigkeiten (Major et al., Ármannsdóttir et al.) die Präferenzmuster der Nutzer (Pett et al.) erklären könnten. Unterschiedliche Aufgaben erfordern unterschiedliche Bewegungen und Kräfte, was durch angepasste Steifigkeiten besser unterstützt werden kann (z.B. höhere RoM bei Steigung erfordert eventuell weichere Steifigkeit).
◦
Implikationen für verstellbare Prothesen: Argumentieren Sie stark auf Basis der Ergebnisse von Pett et al., dass die signifikante Variabilität der bevorzugten Steifigkeit über Aktivitäten hinweg die Notwendigkeit für Prothesen mit anpassbarer oder variabler Steifigkeit unterstreicht. Eine feste Steifigkeit kann nicht optimal für alle Aktivitäten sein. Variable Steifigkeit könnte es Nutzern ermöglichen, die Prothese an die Anforderungen der spezifischen Aufgabe anzupassen, was potenziell Gangqualität, Komfort und Leistung verbessert.
◦
Patientenpräferenz in der klinischen Anpassung: Diskutieren Sie die Befunde von Shepherd and Rouse im Kontext der klinischen Praxis. Die Diskrepanz in der Präferenz zwischen Prothetisten und Patienten und die höhere Konsistenz der Patienten legen nahe, dass die Rückmeldung des Patienten bei der Anpassung der Steifigkeit eine entscheidende, vielleicht unterschätzte Rolle spielen sollte. Variable Steifigkeit könnte den Anpassungsprozess erleichtern, indem sie Patient und Prothetist ermöglicht, verschiedene Einstellungen effizient zu erkunden.
◦
Einschränkungen der Studien: Reflektieren Sie kritisch die Limitationen der vorgestellten Studien, wie sie von den Autoren genannt werden: kleine Stichprobengrößen, die Verwendung experimenteller Prothesen, deren Eigenschaften von kommerziellen Füßen abweichen können, der Einfluss der Prothesenausrichtung, die in einigen Studien konstant gehalten wurde, die kurze Anpassungszeit an neue Einstellungen oder Geräte, und die spezifischen getesteten Aktivitäten/Steigungen.
◦
Ausblick auf zukünftige Forschung: Leiten Sie aus den Einschränkungen und offenen Fragen die Notwendigkeit weiterer Forschung ab. Vorschläge könnten sein: Studien mit größeren, vielfältigeren Stichproben; Untersuchung eines breiteren Spektrums von Aktivitäten, Geschwindigkeiten und Steifigkeitsbereichen; Langzeitstudien zur Anpassung; die Entwicklung von Methoden zur Identifizierung der wirklich optimalen (nicht nur bevorzugten) Steifigkeit; Untersuchung des Einflusses der Ausrichtung im Zusammenspiel mit variabler Steifigkeit; und die Erprobung nutzersteuerbarer verstellbarer Systeme im Alltag.
