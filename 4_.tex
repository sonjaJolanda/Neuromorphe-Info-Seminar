\section{Schlussfolgerung}
Die Ergebnisse der verschiedenen Studien zeigen, dass die richtige Steifigkeit für einzelne Patienten und Aktivitäten wichtig beim Design einer Prothese ist. Die individuelle Steifigkeit ist der entscheidende Faktor für die Mobilität und die Gesundheit von Patienten mit Unterschenkelamputation, und sie variiert je nach Aktivität. Eine spannende Arbeit beschäftigt sich mit dem Vorhersagen der bevorzugten Steifigkeit mit Hilfe von maschinellen Lernverfahren, mit denen Prothesen ihre Steifigkeit während des Gangs anpassen könnten. Dies ist ein vielversprechender Ansatz, der allerdings noch viel Potenzial für zukünftige Forschung bietet. 

% in research rabbit gab es noch drei Paper die dieses Paper zitieren, die könnte ich mir an sich auch noch ansehen 

% put my opinion in the paper
% es hieß doch irgendwie mal man soll nicht in der Vergangenheit schreiben, aber hier macht Vergangenheit ja irgendwie Sinn?