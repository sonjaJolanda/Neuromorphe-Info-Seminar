\begin{abstract}
Die richtige Steifigkeit bei lower limb Prothesen ist von entscheidender Bedeutung für die Wiederherstellung der Gehfähigkeit und Balance bei Menschen mit Unterschenkelamputation. Eine suboptimale Steifigkeit kann zu Kompensationen und langfristigen Gesundheitsproblemen führen. Die Bestimmung der optimalen Prothesen-Sprunggelenk-Steifigkeit für jeden Nutzer bleibt komplex. Studien untersuchen die Auswirkungen verschiedener Steifigkeiten auf Biomechanik und vergleichen die Präferenzen von Orthopädietechnikern und Prothesenträgern, die oft differieren. Patientenpräferenzen sind konsistent und korrelieren mit Gelenksymmetrie. Bevorzugte Steifigkeit variiert stark nach Aktivität, was variable Prothesen erfordert. Maschinelles Lernen bietet Potential zur automatisierten Präferenzvorhersage. Ziel ist die Verbesserung der Prothesengestaltung zur Leistungsoptimierung.

% das muss ich noch umschreiben, das ist komplett generiert, außerdem hab ich da keine Quellenangabe!

% Type your 100 words abstract here. Please do not modify the style
% of the paper. In particular, keep the text offsets to zero when
% possible (see above in this `SeminarV2.tex' sample file). You may
% \emph{slightly} modify it when necessary, but strictly respecting
% the margin requirements is mandatory (see the instructions to
% authors for more details).
\end{abstract}