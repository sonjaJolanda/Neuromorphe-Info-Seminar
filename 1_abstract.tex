\begin{abstract}
Eine optimale Steifigkeit einer Unterschenkelprothese ist entscheidend für die Mobilität und Gesundheit eines Patienten mit Unterschenkelamputation. Die Vielfalt von Prothesenfüßen und ihre Funktionalitäten ist groß und mittlerweile gibt es zudem Prothesen, die ihre Steifigkeit während des Ganges anpassen können. Diese ist jedoch schwer individuell zu bestimmen. Kliniker nutzen dafür ihre visuelle Beurteilung und das Feedback der Patienten. Dies ist jedoch aufwendig und für eine Anwendung in motorisierten Prothesen ungeeignet. Die Nutzung maschineller Lernverfahren kann helfen. Insbesondere mit nutzerspezifischen Daten können genaue Vorhersagen zur bevorzugten Steifigkeit einer Prothese gemacht werden.
% Eine optimale Steifigkeit einer Unterschenkelprothese ist entscheidend für die Mobilität und Gesundheit eines Patienten mit Unterschenkelamputation \cite{Shepherd.2020}. Die Vielfalt von Prothesenfüßen und ihre Funktionalitäten ist groß \cite{Stevens.2018} und mittlerweile gibt es zudem Prothesen, die ihre Steifigkeit während des Ganges anpassen können \cite{Shepherd.2017}. Diese ist jedoch schwer individuell zu bestimmen \cite{Shetty.2022}. Kliniker nutzen dafür ihre visuelle Beurteilung und das Feedback der Patienten \cite{Shepherd.2017}. Dies ist jedoch aufwendig und für eine Anwendung in motorisierten Prothesen ungeeignet \cite{Shetty.2022}. Die Nutzung maschineller Lernverfahren kann helfen \cite{Shetty.2022}. Insbesondere mit nutzerspezifischen Daten können genaue Vorhersagen zur bevorzugten Steifigkeit einer Prothese gemacht werden \cite{Shetty.2022}.
\end{abstract}