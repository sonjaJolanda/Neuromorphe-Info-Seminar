\begin{abstract}
Eine optimale Steifigkeit einer Unterschenkelprothese ist entscheidend für die Mobilität und Gesundheit von Patienten mit Unterschenkelamputation. Die Vielfalt der Prothesenfüße ist groß, und es gibt mittlerweile sogar Prothesen, die die Steifigkeit während des Gehens anpassen können. Die optimale Steifigkeit ist jedoch individuell und schwer zu bestimmen. Orthopädietechniker nutzen das Feedback ihrer Patienten und ihre visuelle Beurteilung. Dies ist jedoch aufwendig und für eine Anwendung in motorisierten Prothesen ungeeignet. Das Ziel dieser Arbeit ist es, einen Einblick in die Forschung zur optimalen Steifigkeit zu geben. Dabei werden insbesondere die Rolle der Patientenpräferenzen und das Potenzial maschineller Lernverfahren zur Vorhersage dieser Präferenzen betrachtet. 
\end{abstract}