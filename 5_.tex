\section{Schlussfolgerung}
Inhalt: Fassen Sie die wichtigsten Erkenntnisse prägnant zusammen. Wiederholen Sie, dass die Forschung deutlich zeigt, dass eine feste Sprunggelenksteifigkeit in Prothesen die optimale Leistung und den Komfort über das Spektrum der alltäglichen Aktivitäten hinweg einschränkt. Die bevorzugte Steifigkeit variiert signifikant je nach Aufgabe, und die Präferenzen von Nutzern und Prothetisten können unterschiedlich sein. Das Verständnis und die Berücksichtigung der Nutzerpräferenz ist entscheidend für eine erfolgreiche Anpassung. Betonen Sie, dass die Entwicklung und Implementierung von Prothesen mit variabler oder anpassbarer Steifigkeit das Potenzial hat, die Mobilität, den Komfort und letztlich die Lebensqualität von Menschen mit Unterschenkelamputation erheblich zu verbessern. Schließen Sie mit einem Ausblick auf die fortlaufende Relevanz und die Notwendigkeit zukünftiger Forschung in diesem Bereich.