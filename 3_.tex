\section{Forschung zur Steifigkeit von Prothesenfüßen} 
Es ist also wichtig das Design von Prothesen weiter zu erforschen \cite{Major.2014}.

\subsection{XXX}
% was ist das Ergebnis dieses Papers gewesen? 
\textcolor{red}{XTODOX} untersuchten 2014 die Auswirkungen unterschiedlicher Rotationssteifigkeiten im Sprunggelenk einer experimentellen Fuß-Knöchel-Gelenkprothese auf verschiedene Gehparameter \cite{Major.2014}.  % TODO 
Es wurde eine individuell angefertigte experimentelle Prothese, eine CFAM, verwendet, die eine systematische Variation mechanischer Eigenschaften ermöglichte, während andere Eigenschaften konstant bleiben konnten \cite{Major.2014}. 
Die Ergebnisse der Studie deuten darauf hin, dass die Gehleistung bei Prothesen mit geringerer Dorsalextension (Hebung des Fußes in Richtung Schienbein) Steifigkeit profitieren könnte \cite{Major.2014}. % kann ich das in Klammern lassen?
Die geringere Dorsalextension-Steifigkeit führte zu einer größeren maximalen Dorsalextension des Sprunggelenks und zu einer größeren maximalen Beugung des Kniegelenks der gesunden Seite im Stehen \cite{Major.2014}. Außerdem führte sie tendenziell zu einer reduzierten maximalen Bodenreaktionskraft während des Gehens, was potenziell vorteilhaft für die Gelenke und das Restglied sein könnte \cite{Major.2014}.  Zudem führte die geringere Dorsalextension-Steifigkeit zu einem geringeren Energierverbrauch beim Gehen, möglicherweise jedoch ohne klinische Signifikant \cite{Major.2014}.

\subsection{Wahrnehmung der optimalen Prothesensteifigkeit von Orthopädietechnikern und Prothesenträgern}
Eine weitere Studie von \textcolor{red}{XTODOX} aus dem Jahr 2020 konzentriert sich auf die Analyse der Wahrnehmungen von Prothesenträgern und Orthopädietechnikern \cite{Shepherd.2020}. Forscher stützen ihre Empfehlungen auf biometrische Analysen, während Orthopdietechniker sich auf qualitatives Feedback von Patienten und die visuelle Beurteilung des Ganges stützen \cite{Shepherd.2020}. Aus diesem Grund ist eine qualitative hochwertige Analyse der Wahrnehmung wichtig \cite{Shepherd.2020}. Patienten sind in der Lage Faktoren wie Komfort der Fußes, Leichtgängigkeit der Bewegung, Vertrauen in das Gleichgewicht oder auch die lokale Muskelermüdung wahrzunehmen \cite{Shepherd.2020}. Orthopädietechniker können sich neben diesem Feedback auch auf ihre visuelle Einschätzung des Ganges des Patienten und jahrelange Erfahrung stützen \cite{Shepherd.2020}. 
Patienten und Orthopdietechniker hatten in der Regel nicht die gleiche Präferenz \cite{Shepherd.2020}.  Orthopädietechniker bevorzugten eine höhere Steifigkeit als die Prothesenträger, genauer gesagt um plus 26\% \cite{Shepherd.2020}. Das könnte darauf hindeuten, dass die Anzeichen für eine zu geringe Steifigkeit visuell deutlich sind \cite{Shepherd.2020}. 
Die Meinungen der Patienten waren erheblich konsistenter, als die der verschiedenen Orthopädietechniker \cite{Shepherd.2020}. Auch wenn die Studie die optimale Steifigkeit nicht bestimmen kann, kann aus dieser Konsistenz jedoch abgeleitet werden, dass das Feedback der Patienten, wenn sie unterschiedliche Steifigkeiten testen können, sehr präzise ist \cite{Shepherd.2020}.

\subsection{Präferenz der Prothesensteifigkeit und ihre Beziehung zu messbaren Parametern}
Auch \textcolor{red}{XToDoX} erkannten die Relevanz von Patientenpräferenzen für die Prothesensteifigkeit und führten 2021 eine Studie durch, um die Beziehungen zwischen der Präferenz der Patienten für die Prothesensteifigkeit und verschiedenen anthropometrischen, metabolischen, biomechanischen und leistungsbasierten Messgrößen zu untersuchen \cite{Clites.2021}. Die Präferenzen eines Patienten folgen aus dessen physiologischen und biomechanischen Wahrnehmungen, einer Vielzahl an messbaren Informationen \cite{Clites.2021}.
Während der Nutzung des \textbf{Variable Stiffness Prosthetic Ankle (VSPA)}-Fußes, der eine variable Steifigkeit während des Gehens ermöglicht, wurden die Gangbiomechanik und der metabolischer Energieverbrauch auf einem Laufband analysiert \cite{Clites.2021}. \textcolor{red}{XToDoX} konnten einiges feststellen \cite{Clites.2021}. 
Es gibt eine signifikante nichtlineare Beziehung zwischen Laufbandgeschwindigkeit und bevorzugter Steifigkeit \cite{Clites.2021}. Die bevorzugte Steifigkeit ist am niedrigsten bei mittleren Geschwindigkeiten und steigt bei langsamerer und schnellerer Geschwindigkeit \cite{Clites.2021}. 
Schwerere Patienten bevorzugen außerdem keine höheren Steifigkeiten als leichtere Patienten \cite{Clites.2021}.
Es wurde kein signifikanter linearen Zusammenhang zwischen Gewicht und Steifigkeit gefunden \cite{Clites.2021}. 
Der Zusammenhang von Energierverbrauch und Steifigkeit wurde ebenfalls untersucht \cite{Clites.2021}. Auch hier konnte kein signifikanter linearer oder quadratischer Zusammenhang zwischen Steifigkeit und metabolischer Rate gefunden werden \cite{Clites.2021}. Die Laufbandgeschwindigkeit hatte jedoch einen signifikanten linearen Einfluss auf den Energieverbrauch \cite{Clites.2021}.
In der selbst gewählte Laufgeschwindigkeit konnten die Autoren ebenfalls einen Zusammenhang zur Steifigkeitspräferenz feststellen \cite{Clites.2021}. Die Patienten liefen langsamer, wenn die Steifigkeitswerte niedriger waren als bevorzugt \cite{Clites.2021}.
Auch biomechanische Merkmale, wie Gelenkwinkel und Gelenkmomente, wurden untersucht \cite{Clites.2021}. Dabei konnten zehn Merkmale ohne Zusammenhang, neun Merkmale mit linearem Zusammenhang und sechs Merkmale mit quadratischem Zusammenhang zur Steifigkeit gefunden werden \cite{Clites.2021}. Eines dieser Merkmale ist die Root-Mean-Square-Differenz des Sprunggelenkwinkels zwischen beiden Beinen, ein Maß für die Bewegungsasymmetrie des Sprunggelenks \cite{Clites.2021}. Dieses zeigte ein Extremum nahe der bevorzugten Steifigkeit \cite{Clites.2021}.
Daraus kann abgeleitet werden, dass ein Patient die Prothesensteifigkeit auswählt, bei der die Gangbewegung am symmetrischten ist \cite{Clites.2021}.
Patientenpräferenzen sind ein direktes Maß der Patientenwünsche, aber auch ein mit Unsicherheiten behaftetes Maß \cite{Clites.2021}. Die in der Studie identifizierten Korrelationen zwischen Präferenz und biomechanischen und leistungsbezogenen Maßen können genutzt werden, um schnell Design- und Steuerparameter von Fußprothesen auswählen zu können \cite{Clites.2021}.

\subsection{XX}
\cite{Louessard.2022} % dann das Paper auch kurz zusammenfassen 

\subsection{XX}
\cite{Vaca.2022} % dann das Paper auch kurz zusammenfassen

\subsection{XX}
\cite{InstituteofElectricalandElectronicsEngineers.2024} % dann das Paper auch kurz zusammenfassen: Präferierte Steifigkeit über verschiedene Aktivitäten hinweg

\subsection{Das Vorhersagen von Steifigkeitspräferenzen einer Quasi-passiven Prothese}






In den letzten Jahren gab es einige Fortschritte in der Entwicklung von Prothesen \cite{Shetty.2022}. Motorisierte Prothesen versuchen die Defizite herkömmlicher Prothesen zu vermeiden, sind aber teuer, schwer und wenig robust \cite{Shetty.2022}. Quasi-passive Prothesen mit einem kleinen Motor und einer Batterie versuchen einen Kompromiss zwischen der Funktionalität komplexer Systeme und deren Herausforderungen zu finden \cite{Shetty.2022}. 
Einige neue quasi-passive Prothesen können die Steifigkeit von Schritt zu Schritt anpassen um sie an unterschiedliche Aktivitäten, wie Gehen oder Treppensteigen, anzupassen \cite{Shetty.2022}. Die Steifigkeit muss dabei korrekt abgestimmt werden, um den gewünschten Effekt zu erreichen \cite{Shetty.2022}. Diese Aufgabe ist eine noch offene und wichtige Forschungsfrage \cite{Shetty.2022}. Das Konzept der Nutzerpräferenz setzt sich langsam als Steuergröße für assitive Technologien durch \cite{Shetty.2022}. Patienten weisen unterschiedliche, aber konsistente Präferenzen für die Steifigkeit auf \cite{Shetty.2022}. Außerdem konnte gezeigt werden, dass die bevorzugte Steifigkeit die kinematische Symmetrie im Sprunggelenk maximiert \cite{Shetty.2022}. Die Bestimmung der Patientenpräferenzen ist jedoch aufwendig \cite{Shetty.2022}. Aus diesem Grund möchten \textcolor{red}{XTODOX} diese Präferenzen vorhersagen, anstatt sie experimentell zu erheben \cite{Shetty.2022}. Diese Vorhersagen versuchen die Autoren mit Hilfe eines maschinellen Lernmodells zu generieren \cite{Shetty.2022}. Dieses soll aus biomechanischen Daten Merkmale herausfiltern, aus denen Nutzerpräferenzen abgeleitet werden können \cite{Shetty.2022}. 

Die Autoren implementierten zwei klassische maschinelle Lernverfahren und drei Deep Learning Verfahren und verglichen sie miteinander \cite{Shetty.2022}. % hier noch genauer beschreiben 

Das Ergebnis dieser Studie ist die Erkenntnis, dass biomechanische Daten mit Deep Learning Modellen effektiv genutzt werden können, um Nutzerpräferenzen mit nutzerspezifischen Trainingsdaten vorhersagen zu können \cite{Shetty.2022}. Damit ergibt sich eine neue, einfachere und praktikablere Methode um die bevorzugte individuelle Prothesensteifigkeit von Patienten zu ermitteln \cite{Shetty.2022}. 