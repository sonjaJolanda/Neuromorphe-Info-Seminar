\section{Forschungsansätze und wichtige Ergebnisse}
VSPA Foot (Variable-Stiffness Prosthetic Ankle-Foot) *   Beschreibung und Funktionsweise (Blattfeder, Nockengetriebe, verstellbare Auflage/Schieber) *   Steifigkeitsbereich (200 bis 1060 Nm/rad mit linearer Drehmoment-Winkel-Kurve) *   Verhältnis Plantarflexions- zu Dorsiflexionssteifigkeit (1:3) *   Stellen der Steifigkeit während der Schwungphase *   Vorteile für Nutzerstudien (nahtloses Erleben verschiedener Steifigkeiten, schnelles Finden von Präferenzen, leichtgewichtig) *   Teilnehmer (Anzahl, Profil, Einschlusskriterien) *   Akklimatisierung an die Laborumgebung und die Prothese *   Verfahren zur Bestimmung der präferierten Steifigkeit *   Bedienung über ein elektronisches Drehrad *   Durchführung über fünf Aktivitäten (Ebene, Neigung aufwärts/abwärts, Treppe aufwärts/abwärts) *   Feste Reihenfolge der Aktivitäten aus praktischen Gründen *   Selbstausgewählte Gehgeschwindigkeit auf dem Laufband *   Eigenes Tempo auf der Treppe *   Sieben Präferenz-Identifikationsversuche pro Aktivität *   Startsteifigkeit und Re-Seeding *   Exploration des vollen Steifigkeitsbereichs *   Beschreibung der Lokomotionsaktivitäten *   Ebenerdiges Gehen (Laufband) *   Gehen auf Neigung/Gefälle (Laufband, 4,7º) *   Treppe aufwärts/abwärts (statisches Treppenhaus, Beschreibung der Stufenmaße) *   Nutzung von Handläufen *   Messtechnik *   IMU zur Gangerkennungsphasenschätzung *   Motor zur Steifigkeitsanpassung *   Winkelencoder für Drehradposition und Knöchelkinematik *   Onboard-Computer und Elektronik *   Datenerfassung und -verarbeitung
 
Präferierte Steifigkeit *   Unterschiede der durchschnittlichen präferierten Steifigkeit zwischen den Aktivitäten *   Quantitative Ergebnisse (Nm/rad) für jede Aktivität *   Präferenzen relativ zum ebenen Gehen (\%) *   Unterschiede übersteigen die Wahrnehmungsschwelle (7,7\%) *   Große Variation der Präferenzen zwischen den Teilnehmern innerhalb einer Aktivität *   Geringere Variation innerhalb der Teilnehmer für eine bestimmte Aktivität *   Konsistenz der Präferenzauswahl variiert mit der Aktivität *   Kinematik und Gehgeschwindigkeit *   Unterschiede in der Knöchelkinematik zwischen den Aktivitäten *   Mittlere Spitzendorsalextension für jede Aktivität *   Inverse Beziehung zwischen Prothesensteifigkeit und Bewegungsbereich des Knöchels *   Selbstausgewählte Gehgeschwindigkeiten für Laufbandaktivitäten

Inhalt: Dieses Kapitel ist das Herzstück Ihrer Arbeit, in dem Sie die spezifischen Studien zusammenfassen. Teilen Sie es thematisch auf, um die verschiedenen Aspekte der Forschung zur Steifigkeit darzustellen.
◦
Systematische Untersuchung fester Steifigkeiten (Major et al.): Beschreiben Sie den Ansatz von Major et al., die Effekte fester (niedriger und hoher) Dorsalextension- und Plantarflexionssteifigkeiten auf Gangkinematik, Prothesenbelastung und metabolische Kosten systematisch mittels eines experimentellen Prothesenfußes (CFAM) untersuchten. Fassen Sie die wichtigsten Ergebnisse zusammen: Niedrige Dorsalextensionsteifigkeit führte im Allgemeinen zu größerer Dorsalextensionsbewegung der Prothesenseite, größerer Kniebeugung auf der gesunden Seite, reduzierter Bodenreaktionskraft während der Belastungsphase und reduzierten metabolischen Kosten. Weisen Sie darauf hin, dass die Unterschiede bei Plantarflexionssteifigkeit gering waren und dass die beobachteten Unterschiede, obwohl tendenziell vorteilhaft bei niedriger Dorsalextension, oft klein waren. Erwähnen Sie, dass niedrigere Dorsalextension die tibiale Progression im späten Stand erleichtern könnte.

Quellen:
Taskabhängige biomechanische Effekte verstellbarer Steifigkeit (Ármannsdóttir et al.): Stellen Sie die Studie von Ármannsdóttir et al. vor, die einen neuartigen Prothesenfuß mit verstellbarer Steifigkeit (VSA Fuß) verwendete, um die Effekte auf die Biomechanik (insbesondere Sprunggelenk RoM und Dynamic Joint Stiffness) bei verschiedenen Gangaufgaben (Level, Steigung, Gefälle, verschiedene Geschwindigkeiten) zu untersuchen. Beschreiben Sie, dass die Dorsalextensionswinkel mit weicherem Fuß und höherer Geschwindigkeit/stärkerer Steigung zunahmen. Heben Sie hervor, dass die Effekte der Steifigkeit auf die Prothesendynamik aufgabenabhängig sind, insbesondere bei kinetischen Parametern. Die Ergebnisse deuten darauf hin, dass ein Fuß, dessen Steifigkeit vom Nutzer an die Aufgabe angepasst werden kann, für aktive Personen vorteilhaft sein könnte.

Quellen:
Präferenzunterschiede zwischen Nutzern und Prothetisten (Shepherd and Rouse): Fassen Sie die Studie von Shepherd and Rouse zusammen, die die Präferenzen für Sprunggelenksteifigkeit bei Nutzern und Prothetisten verglich. Das Hauptresultat hierbei war, dass Prothetisten im Durchschnitt eine signifikant höhere Steifigkeit bevorzugten (um 26\%) als die Patienten. Wichtig ist auch die Erkenntnis, dass Patienten deutlich konsistenter in ihrer Präferenz waren als Prothetisten. Diskutieren Sie kurz mögliche Gründe für diese Diskrepanz (z.B. Prothetisten verlassen sich auf visuelle Hinweise wie "Fusschlag" oder "Drop-off", die eher bei niedriger Steifigkeit auffallen, während Patienten das Gefühl der Belastung oder des Energie-Returns stärker wahrnehmen könnten). Erwähnen Sie, dass bei der gemeinsamen Festlegung einer Steifigkeit (Patient und Prothetist kommunizieren) kein eindeutiger Trend erkennbar war, wer den Prozess stärker beeinflusste.
Quellen:
Aufgabenabhängige Präferenz des Nutzers (Pett et al.): Präsentieren Sie ausführlich die Ergebnisse von Pett et al., da dies Ihre Hauptreferenz ist. Beschreiben Sie, dass diese Studie die bevorzugte Sprunggelenksteifigkeit von transtibial amputierten Nutzern über fünf verschiedene Aktivitäten quantifizierte: Level Walking, Steigung, Gefälle, Treppe aufwärts, Treppe abwärts. Das Schlüsselergebnis ist, dass die bevorzugte Steifigkeit erheblich zwischen den Aktivitäten variierte. Geben Sie Beispiele für die Unterschiede an, z.B. dass die Präferenzen zwischen Gefälle-Gehen und Treppe abwärts um 31,8\% der Präferenz für Level Walking differierten. Betonen Sie, dass diese Unterschiede größer waren als die wahrnehmbare Schwelle (Just-Noticeable Difference) und mehreren "Kategorien" kommerzieller Prothesenfüße entsprachen. Stellen Sie die Beziehung zwischen Steifigkeit und Kinematik dar, z.B. dass die Sprunggelenk-Bewegungsamplitude (RoM) umgekehrt zur Steifigkeit war.