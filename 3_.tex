\section{Forschung zur Steifigkeit von Prothesenfüßen}
Es ist wichtig das Design von Prothesen weiter zu erforschen, um zu verstehen, wie Prothesen gestaltet werden können, sodass die negativen Auswirkungen einer Unterschenkelamputation möglichst gering bleiben \cite{Major.2014}.

\subsection{Auswirkungen verschiedener Prothesensteifigkeiten auf das Gleichgewicht und die Gangbiomechanik}
Im Rahmen einer Studie wurde die Stand- und Gehfähigkeit von Patienten mit Unterschenkelamputationen mit unterschiedlichen Steifigkeiten im Prothesengelenk untersucht \cite{Vaca.2022}. Zur Durchführung der Studie wurde ein Venture-Fuß verwendet, der drei verschiedenen Steifigkeiten durch den Austausch spezifischer Elemente ermöglicht \cite{Vaca.2022}. Die Ergebnisse deuteten auf eine verbesserte Balance als Folge einer höheren Steifigkeit hin \cite{Vaca.2022}. Die Variation der Steifigkeit hatte außerdem signifikante Auswirkungen auf verschiedene Aspekte der Gangbiomechanik, wie die Schrittlänge, die Schrittlängenasymmetrie, die Kinematik und Kinetik und die vertikalen Bodenreaktionskräfte \cite{Vaca.2022}. Diese Erkenntnisse unterstreichen die Notwendigkeit einer Optimierung der Prothesensteifigkeit für die individuellen Bedürfnisse der Patienten \cite{Vaca.2022}.

% \subsection{XXX}
% CFAM erklären, falls ich das paper dazu drinnen lasse
% % was ist das Ergebnis dieses Papers gewesen? 
% 2014 wurden die Auswirkungen unterschiedlicher Rotationssteifigkeiten im Sprunggelenk einer experimentellen Fuß-Knöchel-Gelenkprothese auf verschiedene Gehparameter untersucht \cite{Major.2014}.
% Es wurde eine individuell angefertigte experimentelle Prothese, eine CFAM, verwendet, die eine systematische Variation mechanischer Eigenschaften ermöglichte, während andere Eigenschaften konstant bleiben konnten \cite{Major.2014}. 
% Die Ergebnisse der Studie deuten darauf hin, dass die Gehleistung bei Prothesen mit geringerer Dorsalextension (Hebung des Fußes in Richtung Schienbein) Steifigkeit profitieren könnte \cite{Major.2014}. % kann ich das in Klammern lassen?
% Die geringere Dorsalextension-Steifigkeit führte zu einer größeren maximalen Dorsalextension des Sprunggelenks und zu einer größeren maximalen Beugung des Kniegelenks der gesunden Seite im Stehen \cite{Major.2014}. Außerdem führte sie tendenziell zu einer reduzierten maximalen Bodenreaktionskraft während des Gehens, was potenziell vorteilhaft für die Gelenke und das Restglied sein könnte \cite{Major.2014}.  Zudem führte die geringere Dorsalextension-Steifigkeit zu einem geringeren Energierverbrauch beim Gehen, möglicherweise jedoch ohne klinische Signifikant \cite{Major.2014}.

\subsection{Wahrnehmung der optimalen Prothesensteifigkeit} % von Orthopädietechnikern und Prothesenträgern
Eine weitere Studie konzentriert sich auf die Analyse der Wahrnehmungen von Prothesenträgern und Orthopädietechnikern \cite{Shepherd.2020}. Forscher stützen ihre Empfehlungen auf biometrische Analysen, während Orthopädietechniker auf qualitatives Feedback von Patienten und ihre visuelle Beurteilung des Gangs zurückgreifen \cite{Shepherd.2020}. Patienten sind in der Lage, Faktoren wie Komfort des Fußes, Leichtgängigkeit der Bewegung, Vertrauen in das Gleichgewicht oder auch die lokale Muskelermüdung wahrzunehmen \cite{Shepherd.2020}. Orthopädietechniker können dieses Feedback mit ihrer visuellen Einschätzung des Gangs des Patienten und jahrelanger Erfahrung kombinieren \cite{Shepherd.2020}. Eine hochwertige Analyse dieser Wahrnehmungen ist wichtig \cite{Shepherd.2020}. Die Autoren ließen Patienten mit Unterschenkelamputation mit einer Prothese variabler Steifigkeit gehen, während Orthopädietechniker ihren Gang beobachteten \cite{Shepherd.2020}. Beide Gruppen gaben danach unabhängig voneinander ihre bevorzugte Steifigkeit an \cite{Shepherd.2020}.
Patienten und Orthopädietechniker hatten in der Regel nicht die gleiche Präferenz \cite{Shepherd.2020}. Orthopädietechniker bevorzugten eine höhere Steifigkeit als die Prothesenträger, genauer gesagt um plus 26\% \cite{Shepherd.2020}. Das könnte darauf hindeuten, dass die Anzeichen für eine zu geringe Steifigkeit visuell deutlich sind \cite{Shepherd.2020}.
Die Meinungen der Patienten waren erheblich konsistenter als die der verschiedenen Orthopädietechniker \cite{Shepherd.2020}. Auch wenn die Studie die optimale Steifigkeit nicht bestimmen kann, kann aus dieser Konsistenz abgeleitet werden, dass das Feedback der Patienten, wenn sie unterschiedliche Steifigkeiten testen können, sehr präzise ist \cite{Shepherd.2020}.

\subsection{Präferenz der Prothesensteifigkeit und ihre Beziehung zu messbaren Parametern}
Auch Clites et al. erkannten die Relevanz von Patientenpräferenzen für die Ermittlung der optimalen Prothesensteifigkeit und führten eine Studie durch, um die Beziehungen zwischen der Steifigkeitspräferenz der Patienten und verschiedenen Messgrößen zu untersuchen \cite{Clites.2021}. Während der Nutzung eines VSPA-Fußes, der eine variable Steifigkeit während des Gehens ermöglicht, wurden die Gangbiomechanik und der metabolischer Energieverbrauch auf einem Laufband analysiert \cite{Clites.2021}.
Die Autoren konnten eine signifikante nichtlineare Beziehung zwischen Laufbandgeschwindigkeit und bevorzugter Steifigkeit entdecken \cite{Clites.2021}. Die bevorzugte Steifigkeit war am niedrigsten bei mittleren Geschwindigkeiten und stieg bei langsamerer und schnellerer Geschwindigkeit \cite{Clites.2021}. Es wurde außerdem kein signifikanter linearer oder quadratischer Zusammenhang zwischen Gewicht oder Energieverbrauch und bevorzugter Steifigkeit gefunden \cite{Clites.2021}.
In der selbst gewählten Laufgeschwindigkeit konnten die Autoren ebenfalls einen Zusammenhang zur Steifigkeitspräferenz feststellen \cite{Clites.2021}. Die Patienten liefen langsamer, wenn die Steifigkeit niedriger war als bevorzugt \cite{Clites.2021}.
Auch biomechanische Merkmale wie Gelenkwinkel und Gelenkmomente wurden untersucht \cite{Clites.2021}. Dabei konnten zehn Merkmale ohne Zusammenhang, neun Merkmale mit linearem Zusammenhang und sechs Merkmale mit quadratischem Zusammenhang zur Steifigkeit gefunden werden \cite{Clites.2021}. Eines dieser Merkmale ist die Root-Mean-Square-Differenz des Sprunggelenkwinkels zwischen beiden Beinen, ein Maß für die Bewegungsasymmetrie des Sprunggelenks \cite{Clites.2021}. Dieses zeigte ein Extremum nahe der bevorzugten Steifigkeit \cite{Clites.2021}.
Daraus kann abgeleitet werden, dass ein Patient die Prothesensteifigkeit auswählt, bei der die Gangbewegung am symmetrischsten ist \cite{Clites.2021}.
Patientenpräferenzen sind ein direktes Maß der Patientenwünsche, aber auch ein mit Unsicherheiten behaftetes Maß \cite{Clites.2021}. Die in der Studie identifizierten Korrelationen zwischen Präferenz und messbaren Werten können genutzt werden, um Design- und Steuerparameter von Fußprothesen wählen zu können \cite{Clites.2021}.

\subsection{Bevorzugte Steifigkeit über verschiedene Aktivitäten hinweg}
In den meisten Studien zur Prothesensteifigkeit wird diese nur während des Gehens untersucht \cite{Pett.2024}. Eine Studie aus dem Jahr 2024 erforscht die bevorzugte Steifigkeit von Prothesenfüßen in fünf verschiedenen Aktivitäten, auch hier mit Hilfe des VSPA-Fußes \cite{Pett.2024}. Vier Studienteilnehmer liefen mit der Prothese auf ebenerdigen Boden, bergauf und bergab auf eine Rampe und Treppen hoch und herunter \cite{Pett.2024}. Die bevorzugte Steifigkeit variierte stark zwischen den Aktivitäten \cite{Pett.2024}. Es gab eine durchschnittliche Abweichung von 31.8\% der bevorzugten Steifigkeit für ebenes Gehen \cite{Pett.2024}. Dies macht deutlich, dass Prothesen mit variabler Steifigkeit, die ihr mechanisches Verhalten den verschiedenen Aktivitäten anpassen können, notwendig sind, um die Anforderungen der Patienten zu erfüllen \cite{Pett.2024}.

\subsection{Das Vorhersagen von Steifigkeitspräferenzen}
Einige neue quasi-passive Prothesen können ihre Steifigkeit von Schritt zu Schritt anpassen, um sie auf unterschiedliche Aktivitäten wie das Gehen oder das Treppensteigen, einzustellen \cite{Shetty.2022}. Die Steifigkeit muss dabei korrekt eingestellt werden, um den gewünschten Effekt zu erreichen \cite{Shetty.2022}. Es konnte gezeigt werden, dass die bevorzugte Steifigkeit die kinematische Symmetrie im Sprunggelenk maximiert \cite{Shetty.2022}. Außerdem ist die Nutzerpräferenz eine wichtige Steuergröße \cite{Shetty.2022}. Patienten weisen unterschiedliche, aber konsistente Präferenzen für die Steifigkeit auf \cite{Shetty.2022}. Die Bestimmung der Patientenpräferenzen ist jedoch aufwendig, und aus diesem Grund möchten Shetty et al. diese Präferenzen vorhersagen, anstatt sie experimentell zu erheben \cite{Shetty.2022}. Dazu nutzen die Autoren maschinelle Lernverfahren \cite{Shetty.2022}. Diese sollen aus biomechanischen Daten Merkmale herausfiltern, aus denen Nutzerpräferenzen abgeleitet werden können \cite{Shetty.2022}.
Die Autoren implementierten zwei klassische maschinelle Lernverfahren (ML) und drei Deep Learning (DL) Verfahren und verglichen sie miteinander \cite{Shetty.2022}.
Sie analysierten zudem, wie sich das Einbeziehen von subjektspezifischen Daten auf die Steifigkeitsvorhersage auswirkt \cite{Shetty.2022}. Und sie verglichen drei Gruppen von biomechanischen Signalen, um zu verstehen, wie sich die Menge und Art der Daten auf die Schätzgenauigkeit auswirken \cite{Shetty.2022}.
In der Studie gingen sieben Versuchspersonen mit Unterschenkelamputation mit Hilfe eines VSPA-Fußes, der eine variable Steifigkeit ermöglicht, auf einem Laufband \cite{Shetty.2022}. Dabei ermittelten die Patienten die bevorzugte Steifigkeit mit Hilfe eines Drehknopfes selbstständig, bevor biomechanische Daten für unterschiedliche Steifigkeits- und Geschwindigkeitskombinationen gesammelt wurden \cite{Shetty.2022}. Die Versuchspersonen erarbeiteten außerdem in ihrer bevorzugten, einer schnelleren und einer langsameren Geschwindigkeit ihre bevorzugte Steifigkeit \cite{Shetty.2022}. Zur Erfassung der biomechanischen Daten wurden eine optische Bewegungserfassung und eine Kraftmessplatte verwendet \cite{Shetty.2022}.

Die biomechanischen Daten wurden nachbearbeitet und analysiert, bevor die Algorithmen angewendet wurden \cite{Shetty.2022}. Die Daten wurden außerdem, neben der Aufteilung in Trainingsdaten, Validierungsdaten und Testdaten, auch in Subjekt-unabhängige Daten und Subjekt-abhängige Daten unterteilt \cite{Shetty.2022}. Mit Hilfe einer Gittersuche und der Validierungsmenge wurden die Hyperparameter aller Algorithmen vor dem Training abgestimmt, um die Quadratwurzel des mittleren quadratischen Fehlers (RMSE) zwischen der vorhergesagten und der tatsächlichen bevorzugten Steifigkeit zu minimieren \cite{Shetty.2022}.
Es wurden zwei klassische ML-Algorithmen implementiert \cite{Shetty.2022}. Der K-Nearest-Neighbour-Algorithmus (KNN) macht Vorhersagen basierend auf den Ähnlichkeiten der Daten, bzw. auf Grundlage des gewichteten Durchschnitts der Abstandsfunktion für die K nächsten Punkte \cite{Shetty.2022}. Es wurde die euklidische Abstandsfunktion und ein Nachbarschaftswert von K=5 gewählt \cite{Shetty.2022}. Die Support-Vector-Regression (SVR) bildet die Daten auf eine höhere Dimension ab und sucht dort eine Hyperebene mit optimalen Rändern \cite{Shetty.2022}. Außerdem wurden drei DL-Algorithmen mit der ReLU-Aktivierungsfunktion implementiert \cite{Shetty.2022}. Ein künstliches neuronales Netz (ANN) aus mehreren vollständig verbundenen Schichten ist der erste implementierte DL-Algorithmus \cite{Shetty.2022}. Für den zweiten Algorithmus, ein Convolutional Neural Network (CNN), wurden die Eingabedaten in ein 3D-Format umgewandelt \cite{Shetty.2022}. Der dritte Algorithmus ist ein Long-Short-Term-Memory (LSTM) Netzwerk \cite{Shetty.2022}.
Die Algorithmen hatten einen signifikanten Einfluss auf den RMSE der Vorhersagen \cite{Shetty.2022}. Der RMSE mit den nutzerspezifischen Daten war um 67\% niedriger. Die drei DL-Algorithmen hatten einen niedrigeren RMSE als die beiden klassischen ML-Algorithmen, die DL-Algorithmen waren also besser \cite{Shetty.2022}. Der LSTM hatte den niedrigsten RMSE, $13,4\% \pm7,9\%$ und KNN hatte den höchsten RMSE mit $18,3\% \pm6,0\%$ \cite{Shetty.2022}. Die durchschnittliche Zeit zum Generieren der Vorhersage der Methoden dieser Arbeit war $1,99 \pm 2,22 ms$ \cite{Shetty.2022}.

Das Ergebnis dieser Studie ist die Erkenntnis, dass biomechanische Daten mit Deep-Learning Modellen effektiv genutzt werden können, um Nutzerpräferenzen mit Trainingsdaten vorhersagen zu können \cite{Shetty.2022}. Das Einbeziehen nutzerspezifischer Trainingsdaten verbesserte die Schätzungen der bevorzugten Steifigkeit des Nutzers signifikant \cite{Shetty.2022}. Damit ergibt sich eine neue, einfachere und praktikablere Methode, um die bevorzugte individuelle Prothesensteifigkeit von Patienten zu ermitteln \cite{Shetty.2022}. Die Ansätze dieser Arbeit sind vielversprechend für das Design und die Abstimmung von Roboterprothesen, sind aber durch den Zeitaufwand für die Vorhersagen limitiert \cite{Shetty.2022}. Zukünftige Arbeiten mit größeren Datensätzen könnten die Leistung und die Generalisierbarkeit der Modelle verbessern \cite{Shetty.2022}. Außerdem ist diese Studie auf die Aktivität Gehen beschränkt und die Untersuchung anderer Aktivitäten ist noch offen \cite{Shetty.2022}.

% auch noch das Paper machen?: Armannsd: file:///C:/Users/sonja/IdeaProjects/Neuromorphe-Info-Seminar/Quellen/1-s2.0-S0268003321002060-main.pdf
% Task dependent changes in mechanical and biomechanical measures result from manipulating stiffness settings in a prosthetic foot 